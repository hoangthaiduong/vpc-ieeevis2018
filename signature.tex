\section{Stream signatures and hybrid streams}

\subsection{Hybrid streams}

To do so, we construct \emph{hybrid} streams that combines two given streams (the relative order
matters). For example, the \emph{hybrid isocontour-rmse} combines \emph{isocontour-optimized} and
\emph{rmse-optimized} in a way such that within a region, it has the same chunk ordering as that of
\emph{rmse-optimized}'s, but globally, it visits the regions in the same order that
\emph{isocontour-optimized} does. Another hybrid stream that can be formed is \emph{hybrid
rmse-isocontour}, which is form similarly to how \emph{hybrid isocontour-rmse} is formed, but with
its two input streams switching roles. Since a hybrid stream is designed to retain the
characteristics of the first stream locally (within a region), it can be used as a proxy of the
second stream for comparison against the first. The \emph{hybrid isocontour-rmse} stream can be
compared with \emph{isocontour-optimized}, and the \emph{hybrid rmse-isocontour} stream can be
compared with \emph{rmse-optimized}. Figure \ref{fig:isocontour-plots} provides these comparisons.

The algorithm used to compute a hybrid stream from two input streams is presented below (TODO: placeholder).
\begin{algorithm}
  \KwData{this text}
  \KwResult{how to write algorithm with \LaTeX2e }
  initialization\;
  \While{not at end of this document}{
   read current\;
   \eIf{understand}{
    go to next section\;
    current section becomes this one\;
    }{
    go back to the beginning of current section\;
   }
  }
  \caption{How to write algorithms}
\end{algorithm}
