\section{Data dependent task-optimized streams}

In previous section we demonstrated that different tasks may need a different stream. For example, we saw
that PSNR stream significantly underperforms the histogram stream when applied to the histogram query.
Unfortunately, it is unclear what a best stream is for a given query. Moreover, it is common to stop analysis
task when the result is good enough, thus avoiding streaming all the data. Therefore, we define the best stream as
a sequence of refinements that reach the mininmum error at given stopping bit budget. Alas, in interactive application
we can only make assumptions what will be the number of bits streamed before user decides to stop the streaming. For
example, if a user drasticly changes viewpoint in a volume rendering application, the stream starts almost from scratch.

Taking the lack of control over the stopping budget to the extreme, the best stream becomes the one which minimizes
the error at all bit budgets. However, as in any global optimization problem we may get stuck at local minimum, as
chunks that improve the error but have impact on later refinement will have lower priority.

Given above definition we can compute the optimal stream. It turns out that finding a permutation that minimizes
the total error given a bit budget is exponential in number of chunks. We thus use more practical greedy algorithm (is there
any approximation ratio?) that at each decision point one by one applies remaining chunks and computes the error
difference as if the chunk was not applied. We minimize the local minimum problem by taking the absolute value
of this difference which helps us to avoid a plateaus in the error curve.

\paragraph*{Coarse-to-fine greedy algorithm} starts with no data and the initial error is computed with respect
to the full data set. Then it takes a list of all chunks in the dataset, computes the error as if the chunk was
enabled, and picks the chunk with the highest absolute difference in the error with respect to the current error.
We use absolute difference to avoid the case where the error difference is zero or negative, which would result
in a long stream of chunk that do not decrease the error significantly. This assumption reflects the expectation
of more data meaning better result. The running time of this algorithm is $O(n^3)$ as we start with $n$ chunks
and at each streaming step decrease the chunk count by one. The cube factor comes from the need to perform inverse
wavelet transform and compute the error for each chunk.





Given the error metric we can now construct a baseline stream. Unfortunately, the streamed chunks are unlikely
to be disjoint and thus affect later chunks' errors. This dependency requires to find permutation with the smallest
total error, a problem that is exponential in number of chunks. 

\paragraph*{Fine-to-coarse Algorithm}

As optimization, we perform the error computation for each chunk only {\em once} and then sort these chunks
by the error. This approximation reduces the running time to $O(n \log n)$ and thus makes it more practical if
the stream order needs to be computed for a dataset, for example before the data is streamed over network.



Scheme3 - coarse-to-fine, greedy optimal (more realistic approach)


Here we discuss the problem of finding the ``best'' stream for the analysis task at hand. This problem is hard because it is hard to define the notion of ``best''. We will attempt at defining ``best'' in terms of minimizing the error with each bit read. However actually programming the optimization this way will result in a sub-optimal stream because it is necessary to sometimes lose in the short term to then gain more later on. This fact means that we can't optimize for bit budget A and then claim that that solution is also an optimal sub-solution for budget B > A.

Then our next attempt is another $O(n^2)$ scheme where we attempt to maximize the error difference with each bit (Scheme1)

Detailed Algorithm for Scheme1.

Another way to optimize is the $O(n)$ scheme where we start with every bit and then turn each off and measure the impact on error. Then we sort every bit by its impact (Scheme2).

Detailed Algorithm for Scheme2.

A supposedly better scheme is $O(n^2)$ but going from fine to coarse. We start with everything, then turn one chunk off at a time, pick the most important chunk and remove it, then repeat the process for the rest (Scheme3).

Detailed Algorithm for Scheme3.

We show a figure comparing Scheme1, Scheme2, and Scheme3 on PSNR, histogram, and isocontour for two datasets.
