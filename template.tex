%\documentclass[journal]{vgtc}                % final (journal style)
\documentclass[review,journal]{vgtc}         % review (journal style)
%\documentclass[widereview]{vgtc}             % wide-spaced review
%\documentclass[preprint,journal]{vgtc}       % preprint (journal style)

%% Uncomment one of the lines above depending on where your paper is
%% in the conference process. ``review'' and ``widereview'' are for review
%% submission, ``preprint'' is for pre-publication, and the final version
%% doesn't use a specific qualifier.

%% Please use one of the ``review'' options in combination with the
%% assigned online id (see below) ONLY if your paper uses a double blind
%% review process. Some conferences, like IEEE Vis and InfoVis, have NOT
%% in the past.

%% Please note that the use of figures other than the optional teaser is not permitted on the first page
%% of the journal version.  Figures should begin on the second page and be
%% in CMYK or Grey scale format, otherwise, colour shifting may occur
%% during the printing process.  Papers submitted with figures other than the optional teaser on the
%% first page will be refused. Also, the teaser figure should only have the
%% width of the abstract as the template enforces it.

%% These few lines make a distinction between latex and pdflatex calls and they
%% bring in essential packages for graphics and font handling.
%% Note that due to the \DeclareGraphicsExtensions{} call it is no longer necessary
%% to provide the the path and extension of a graphics file:
%% \includegraphics{diamondrule} is completely sufficient.
%%
\ifpdf%                                % if we use pdflatex
  \pdfoutput=1\relax                   % create PDFs from pdfLaTeX
  \pdfcompresslevel=9                  % PDF Compression
  \pdfoptionpdfminorversion=7          % create PDF 1.7
  \ExecuteOptions{pdftex}
  \usepackage{graphicx}                % allow us to embed graphics
                                % files
  \DeclareGraphicsExtensions{.pdf,.png,.jpg,.jpeg} % for pdflatex we expect .pdf, .png, or .jpg files
\else%                                 % else we use pure latex
  \ExecuteOptions{dvips}
  \usepackage{graphicx}                % allow us to embed graphics files
  \DeclareGraphicsExtensions{.eps}     % for pure latex we expect eps files
\fi%

%% it is recomended to use ``\autoref{sec:bla}'' instead of ``Fig.~\ref{sec:bla}''
\graphicspath{{figures/}{pictures/}{images/}{./}{img/}} % where to search for the images

\usepackage{microtype}                 % use micro-typography (slightly more compact, better to read)
\PassOptionsToPackage{warn}{textcomp}  % to address font issues with \textrightarrow
\usepackage{textcomp}                  % use better special symbols
\usepackage{mathptmx}                  % use matching math font
\usepackage{times}                     % we use Times as the main font
\renewcommand*\ttdefault{txtt}         % a nicer typewriter font
\usepackage{cite}                      % needed to automatically sort the references
\usepackage{tabu}                      % only used for the table example
\usepackage{booktabs}                  % only used for the table example
\usepackage{color}
%\usepackage{algorithmic}
% \usepackage[]{algorithm2e}
%\usepackage[algo2e]{algorithm2e} 
\usepackage{amsmath}
\usepackage{amsfonts}
\usepackage{algorithm,algpseudocode}
%\usepackage{algpseudocode}
%\usepackage{algorithm}
\usepackage[dvipsnames]{xcolor}
\usepackage{xspace}
\algnewcommand{\Inputs}[1]{%
  \State \textbf{Inputs:}
  \Statex \hspace*{\algorithmicindent}\parbox[t]{.9\linewidth}{\raggedright #1}
}
\algnewcommand{\Output}[1]{%
  \State \textbf{Output:}
  \Statex \hspace*{\algorithmicindent}\parbox[t]{.9\linewidth}{\raggedright #1}
}
\algnewcommand{\Initialize}[1]{%
  \State \textbf{Initialize:}
  \Statex \hspace*{\algorithmicindent}\parbox[t]{.8\linewidth}{\raggedright #1}
}
%%\newcommand{\norm}[1]{\left\lVert#1\right\rVert}
%% We encourage the use of mathptmx for consistent usage of times font
%% throughout the proceedings. However, if you encounter conflicts
%% with other math-related packages, you may want to disable it.

%% In preprint mode you may define your own headline.
%\preprinttext{To appear in IEEE Transactions on Visualization and Computer Graphics.}

%% If you are submitting a paper to a conference for review with a double
%% blind reviewing process, please replace the value ``0'' below with your
%% OnlineID. Otherwise, you may safely leave it at ``0''.
\onlineid{1207}

%% declare the category of your paper, only shown in review mode
\vgtccategory{Research}
%% please declare the paper type of your paper to help reviewers, only shown in review mode
%% choices:
%% * algorithm/technique
%% * application/design study
%% * evaluation
%% * system
%% * theory/model
\vgtcpapertype{Evaluation}

%% Paper title.
\title{A Study of the Tradeoff between Reducing Precision and Reducing Resolution for Data Analysis and Visualization}

%% This is how authors are specified in the journal style

%% indicate IEEE Member or Student Member in form indicated below
\author{Duong Hoang, Pavol Klacansky, Harsh Bhatia, Peer-Timo Bremer, Peter Lindstrom, Valerio Pascucci}
\authorfooter{
%% insert punctuation at end of each item
\item
 Hoang, Klacansky, and Pascucci are with the SCI Institute at the University of Utah, USA. E-mail: \{duong, klacansky, pascucci\}@sci.utah.edu
\item
 Bhatia, Bremer, and Lindstrom are with Lawrence Livemore National Laboratory, USA. E-mail: \{hbhatia, bremer5, pl\}@llnl.gov
}

%other entries to be set up for journal
\shortauthortitle{Hoang \MakeLowercase{\textit{et al.}}: Studying the Tradeoff between Reducing Precision and Reducing Resolution for Data Analysis and Visualization}

%% Abstract section.
\abstract{There currently exist two dominant strategies to reduce data sizes in analysis and
  visualization: reducing the {\it precision} of the data, e.g., through compression, or reducing
  its {\it  resolution}, e.g., by subsampling. Both have advantages and disadvantages and both face
  fundamental limits at which the reduced information ceases to be useful. The paper explores the
  additional gains that could be achieved by combining both strategies. In particular, we present a
  common framework that allows us to study the trade-off in reducing precision and/or resolution in
  a principled manner. We represent data reduction schemes as a progressive stream of bits, and
  study how various bit orderings such as by resolution, by precision, etc. impact the resulting
  approximation error across a wide range of test data and analysis tasks. Furthermore, we compute
  streams optimized for different tasks, to serve as lower bounds on the achievable error.
  Scientific data management systems can use the results presented in this paper as guidance on how
  to store and stream data to make efficient use of the limited storage and bandwidth in practice.\vspace{-0.5em}
% Performing scientific analysis on large data commonly found nowadays is an onerous task
% due to the prohibitive cost of data transfer. Currently, this issue is alleviated by working with
% reduced-resolution or reduced-precision data.
% \peter{Mention progressive access here?}
% This paper brings together techniques that reduce data
% in resolution, precision, or both, into a common framework in which they can be studied and
% compared. The error patterns of these techniques are compared in terms of fundamental error metrics on function values (e.g., L2 norm),
% first and second derivatives, histograms, and isocontours. We also compute and study
% metric-dependent streams
% \peter{streams optimized for different metrics?}
% that serve as empirical bounds on the limit to which
% reduction can happen, given an error tolerance for the analysis task at hand. Finally, based on the
% observed characteristics of those streams, we propose practical heuristics to minimize the amount of data
% needed to perform a given analysis task. The insights and heuristics presented here can be
% leveraged to implement data-optimal\pavol{"optimal" or other word?}
% \peter{what does data-optimal representations of data mean?}
% representations and querying systems
% \peter{query systems?  I.e., a noun, not a verb?}
% to facilitate interactive
% exploration as well as cursory analysis of enormous data.%
% \peter{The abstract does not make it clear what it is we're optimizing,
% i.e. ordering of bits.}
} % end of abstract

%% Keywords that describe your work. Will show as 'Index Terms' in journal
%% please capitalize first letter and insert punctuation after last keyword
% \keywords{Radiosity, global illumination, constant time}

%% ACM Computing Classification System (CCS). 
%% See <http://www.acm.org/class/1998/> for details.
%% The ``\CCScat'' command takes four arguments.

\CCScatlist{ % not used in journal version
 \CCScat{K.6.1}{Management of Computing and Information Systems}%
{Project and People Management}{Life Cycle};
 \CCScat{K.7.m}{The Computing Profession}{Miscellaneous}{Ethics}
}

%% Uncomment below to include a teaser figure.
\teaser{
  \centering
  \includegraphics[width=0.99\linewidth]{cats}
  % \subcaptionbox{By level}{
  % {\includegraphics[width=0.15\linewidth]{teaser_rmse_level}}}
  % \subcaptionbox{By bit plane}{
  % {\includegraphics[width=0.15\linewidth]{teaser_rmse_bit_plane}}}
  % \subcaptionbox{By wavelet norm}{
  % {\includegraphics[width=0.15\linewidth]{teaser_rmse_wavelet_norm}}}
  % \subcaptionbox{Signature}{
  % {\includegraphics[width=0.15\linewidth]{teaser_rmse_signature}}}
  % \subcaptionbox{Groundtruth}{
  % {\includegraphics[width=0.15\linewidth]{teaser_rmse_groundtruth}}}
  % \\
  % \subcaptionbox{By level}{
  % {\includegraphics[width=0.15\linewidth]{teaser_laplacian_level}}}
  % \subcaptionbox{By bit plane}{
  % {\includegraphics[width=0.15\linewidth]{teaser_laplacian_bit_plane}}}
  % \subcaptionbox{By wavelet norm}{
  % {\includegraphics[width=0.15\linewidth]{teaser_laplacian_wavelet_norm}}}
  % \subcaptionbox{By signature}{
  % {\includegraphics[width=0.15\linewidth]{teaser_laplacian_signature}}}
  % \subcaptionbox{Groundtruth}{
  % {\includegraphics[width=0.15\linewidth]{teaser_laplacian_groundtruth}}}
  \caption{Renderings of the \emph{diffusivity} field at 0.1 bits per sample (bps), and its Laplacian field, at 0.6 bps, using two of the bit streams studied in the paper. Compared to the \emph{by bit plane} stream, \emph{by wavelet norm} produces a better reconstruction of the original function (left, see white features), but a worse reconstruction of the Laplacian field (right).}
  \label{fig:teaser}
}

%% Uncomment below to disable the manuscript note
%\renewcommand{\manuscriptnotetxt}{}


\newcommand{\dense} {
  \setlength{\itemindent}{0mm}
  \setlength{\leftmargin}{0mm}
  \setlength{\parskip}{0mm}
  \setlength{\itemsep}{1mm}
}

\newcommand{\norm}[1]{\left\lVert#1\right\rVert}
\newcommand{\notetext}[1]{%
  \raggedright\tiny\sffamily\bfseries{#1}%
}
\newcommand{\note}[1]{%
  \ifinner%
    \smash{%
      \raggedright%
      \hspace*{\textwidth}%
      \hspace*{\marginparsep}%
      \parbox[t]{\marginparwidth}{\notetext{#1}}%
    }%
    \\[-0.5\bs]%
  \else%
    \marginpar{\notetext{#1}}%
  \fi%
}
\providecommand{\etal}{et al.\@\xspace}

\usepackage[normalem]{ulem}
\newcommand{\pavol}[1]{{\textcolor{cyan}{(Pavol: #1)}}}
\newcommand{\hb}[1]{{\textcolor{blue}{(HB: #1)}}}
\newcommand{\hbadd}[1]{{\textcolor{blue}{#1}}}
\newcommand{\hbdel}[1]{{\textcolor{red}{Delete: #1}}} % was failing on my machine
%\newcommand{\hbdel}[1]{{\textcolor{red}{}}}
\newcommand{\todo}[1]{{\textcolor{red}{#1}}}
%\newcommand{\duong}[1]{{\textcolor{purple}{(Duong: #1)}}}
\newcommand{\duong}[1]{{\textcolor{purple}{}}} % this disable comments
\newcommand{\peter}[1]{{\textcolor{green}{(PL: #1)}}} % this disable comments
\newcommand{\ptb}[1]{{\textcolor{violet}{(PTB: #1)}}} % this disable comments

% prevent hyphenation
\usepackage[none]{hyphenat}
\sloppy

% ignore figures for fast compilation
% \usepackage[allfiguresdraft]{draftfigure}

\usepackage[mathcal]{euscript}

\newcommand {\mm}[1] {\ifmmode{#1}\else{\mbox{\(#1\)}}\fi}

\newcommand{\Lspace}{{\mathbb L}}
\newcommand{\strm}	{\mm{\mathcal{S}}\xspace}
\newcommand{\sgn}	{\mm{\Lambda}\xspace}
\newcommand{\sopt}  {\mm{\strm_{\text{opt}}}\xspace}
\newcommand{\slvl}  {\mm{\strm_{\text{lvl}}}\xspace}
\newcommand{\sbit}  {\mm{\strm_{\text{bit}}}\xspace}
\newcommand{\swav}  {\mm{\strm_{\text{wav}}}\xspace}
\newcommand{\smag}  {\mm{\strm_{\text{mag}}}\xspace}
\newcommand{\ssig}  {\mm{\strm_{\text{sig}}}\xspace}
\newcommand{\stkop}  {\mm{\strm_{\text{[task]-opt}}}\xspace}
\newcommand{\stksg}  {\mm{\strm_{\text{[task]-sig}}}\xspace}
\newcommand{\srop}  {\mm{\strm_{\text{rmse-opt}}}\xspace}
\newcommand{\srsg}  {\mm{\strm_{\text{rmse-sig}}}\xspace}
\newcommand{\sgop}  {\mm{\strm_{\text{grad-opt}}}\xspace}
\newcommand{\sgsg}  {\mm{\strm_{\text{grad-sig}}}\xspace}
\newcommand{\slop}  {\mm{\strm_{\text{lap-opt}}}\xspace}
\newcommand{\slsg}  {\mm{\strm_{\text{lap-sig}}}\xspace}
\newcommand{\shop}  {\mm{\strm_{\text{hist-opt}}}\xspace}
\newcommand{\shsg}  {\mm{\strm_{\text{hist-sig}}}\xspace}
\newcommand{\siop}  {\mm{\strm_{\text{iso-opt}}}\xspace}
\newcommand{\sisg}  {\mm{\strm_{\text{iso-sig}}}\xspace}

% stream weights
%\newcommand{\wlvl}  {\mm{w_{\text{lvl}}}\xspace}
%\newcommand{\wbit}  {\mm{w_{\text{bit}}}\xspace}
%\newcommand{\wwav}  {\mm{\mathcal{W}_{\text{wav}}}\xspace}
\newcommand{\wwav}  {\mm{w_{\text{wav}}}\xspace}
\newcommand{\wmag}  {\mm{w_{\text{mag}}}\xspace}

% wavelet coeff and basis
\newcommand{\wcof}  {\mm{w}\xspace}
\newcommand{\wbas}  {\mm{\psi}\xspace}
\newcommand{\wsca}  {\mm{\phi}\xspace}

\newcommand{\err}   {\mm{\varepsilon}\xspace}
\newcommand{\x}     {\mm{\mathbf{x}}\xspace}

\newcommand{\para}[1]{\vspace{0.5em}\noindent{\textbf{#1}}}%\hspace{0.04em}}

\usepackage{pifont}% http://ctan.org/pkg/pifont
\newcommand{\cmark}{{\color{ForestGreen}{\ding{51}}}}%
\newcommand{\xmark}{{\color{red}{\ding{55}}}}%

%% Copyright space is enabled by default as required by guidelines.
%% It is disabled by the 'review' option or via the following command:
% \nocopyrightspace

\vgtcinsertpkg
%\usepackage{subcaption}
\usepackage[labelfont=sf]{subcaption}
\captionsetup{labelfont=sf,font=scriptsize,textfont=sf}
\usepackage{cleveref}
\Crefformat{figure}{#2Fig.~#1#3}

%%%%%%%%%%%%%%%%%%%%%%%%%%%%%%%%%%%%%%%%%%%%%%%%%%%%%%%%%%%%%%%%
%%%%%%%%%%%%%%%%%%%%%% START OF THE PAPER %%%%%%%%%%%%%%%%%%%%%%
%%%%%%%%%%%%%%%%%%%%%%%%%%%%%%%%%%%%%%%%%%%%%%%%%%%%%%%%%%%%%%%%%

\begin{document}

%% The ``\maketitle'' command must be the first command after the
%% ``\begin{document}'' command. It prepares and prints the title block.

%% the only exception to this rule is the \firstsection command

\firstsection{Introduction}
\maketitle

\section{Introduction}

Please follow the steps outlined in this document very carefully when
submitting your manuscript to Eurographics.

You may as well use the \LaTeX\ source as a template to typeset your own
paper. In this case we encourage you to also read the \LaTeX\ comments
embedded in the document.


The list of our contributions:

\begin{itemize}
        \item We show that traditional multi-resolution data streaming schemes that group bits either in the same coefficient or in the same bit plane result in suboptimal errors. We reduce error significantly using the same number of bits by allowing bits belonging to the same bit plane or the same coefficient to be streamed separately.

        \item We devise a greedy method to approximate an optimal bit ordering for several quantities of interest, namely RMSE, Histogram, Gradient, Laplacian, Isocontour, Volume rendering. Based on these orderings, we observe that there are more than one optimal ordering for Gradient/Laplacian/Histogram, in particular, the RMSE ordering is also nearly optimal for these quantities.

        \item We show that locally the optimal bit orderings for all tested quantities are just slight variations of the optimal RMSE ordering. Furthermore, locally the (data-dependent) RMSE ordering is approximately the same as a data-independent ordering that can be computed using only the norms of the wavelet basis functions.

        \item We describe a practical algorithm to realize the aforementioned data-independent optimal ordering, taking advantage of an encoding scheme that compress the leading zero bits. We also note that this ordering is globally (as opposed to locally) near optimal in terms of RMSE. 
\end{itemize}

\section{Related work}

Techniques that reduce data in resolution typically build a tree-shape hierarchy over the data. A
very common scheme to genearte such a hierarchy is to construct low-resolution copies of the data
from higher-resolution ones through filtering and subsampling. Examples include Gaussian and
Laplacian pyramids~\cite{laplacian-pyramid},
mipmaps~\cite{multires_octree1999,interactive-exploration-ct-scans}. Often times, the data is stored
in blocks on each resolution level. To save bandwidth, low-resolution blocks can be streamed during
rendering, if the points being queried project to less than a pixel on screen. However, these methods
increase storage requirements, making them unsuitable for very large data.

Recent multi-resolution techniques save storage by adapting to the data in such a way that different
regions of the data are stored in different resolutions, depending on how ``homogeneous'' the region
is. Fast-varying regions are stored at higher resolution. A very popular approach is sparse voxel
octrees (SVO), pioneered by Crassin \etal~\cite{gigavoxels} and Gobbetti \etal~\cite{Gobbetti2008},
and variations of which are found in~\cite{Fogal-2013-RayGuided,visualization-driven}. Sparsity
comes from the fact that smooth-varying regions are stored at coarser octree levels, which
significantly reduces storage. During rendering, blocks of samples are streamed from an appropriate
resolution, determined by how far the queried samples are from the eye/camera. Beyer
\etal~\cite{large‐scale-volume} gives a comprehensive overview of this family of techniques.

Other trees such as B+ tree~\cite{vdb2013} and kd-tree~\cite{fogal-kdtree} can also be used in lieu
of octrees to build a sparse hierarchy. Alternatively, space-filling curves such as the hierarhical
Z curve~\cite{idx2001} can be used to reorder data samples to form a hierarchy without any filtering
steps or redundant samples, as low-resolution levels are constructed via subsampling. Unfortunately,
subsampling can introduce heavy aliasing artifacts.

\duong{continue from here}
A sparse multiresolution hierarchy offers level-of-detail access, which reduces bandwidth. 

wavelets~\cite{treib,multires_toolkit2003,vapor2007,woodring2011}.

For example, the Compression-domain Output-sensitive Volume Rendering Architecture
(COVRA)~\cite{covra2012} constructs a level-of-detail pyramid in a precomputation stage to reduce
memory usage for blocks that are far from the viewpoint. Moreover, COVRA further compresses the
blocks of the pyramid to reduce the data transfer time, either to the GPU or over a network.
However, this redundant level-of-detail representation increases data size, which may be undesirable
for some tasks.

wavelets:
~\cite{treib}
~\cite{multires-framework}
~\cite{compression-domain-volume-rendering}
~\cite{interactive-rendering-large-volume}
~\cite{rapid-compression-volume}
~\cite{survey-multires}
~\cite{multires_toolkit2003}
~\cite{wavelet-compression-interactive-vis}

resolution
~\cite{multires-volume-rendering} (wavelet, level of detail)
~\cite{in-situ-sampling-particle}

tensor:
~\cite{tensor_dvr2015}
~\cite{multiscale-tensor}
~\cite{tamresh}

Octrees:
~\cite{sph-octree} (octree + wavelet compression)

Other trees:

Error analysis:
~\cite{evaluating-compression-climate}
~\cite{compression_sim2013}
~\cite{statistical-volume-quality}
~\cite{evaluating-efficacy-wavelet}
~\cite{topology-verification-isosurface}
~\cite{verifiable-isosurface}
~\cite{verifying-volume-rendering}
~\cite{statistical-volume-quality}

vector quantization:
~\cite{vq1992}
~\cite{hw_dvr2007} (VQ on transfrom domain)
~\cite{compression_domain2003} (VQ after Haar-like transform)
~\cite{covra2012} octrees of bricks with sparse representations

error-guided: 
~\cite{tf_decompression2004} (based on transfer function)

~\cite{spgrid2014}

surveys:
~\cite{state-of-the-art-compressed-volume} (compressed volume rendering)
~\cite{li2018} data reduction techniques

compression:
~\cite{isabela}
~\cite{fpzip}
~\cite{sz}
~\cite{zfp2014}

precision:
~\cite{ezw}
~\cite{spiht1996}
~\cite{mloc}
~\cite{sbhp2000}
~\cite{jpeg2001}

The wavelet transform constructs a hierarchy of resolution levels via low and high bandpass filters.
The transform is recursively applied to the lower-resolution band, resulting in a hierarchy of
``details'' at varying resolution. One benefit of wavelets over redundant representations like
Laplacian pyramids is that the wavelet transform is merely a change of basis that like reordering
techniques does not increase the data size. Another benefit over AMR and other tree-like techniques
is that the wavelet basis functions are defined everywhere in space, requiring no special
interpolation rules when given some arbitrary subset of wavelet coefficients and basis functions.
One disadvantage of the wavelet transform is non-constant time random access cost, though
acceleration structures have been proposed to speed up local queries~\cite{weiss}.

Spatial adaptivity in resolution can be achieved by tiling the wavelet coefficients of individual
subbands. For example, the Visualization and Analysis Platform for Ocean, Atmosphere, and Solar
Researchers (VAPOR) toolkit~\cite{multires_toolkit2003, vapor2007} incorporates a multiresolution
file format based on a wavelets to allow data analysis on commodity hardware, and stores individual
tiles in separate files to allow loading of the region of interest. However, the authors only
leverage the resolution control without exploring the precision axis, which can potentially further
reduce data transfer.

Reducing precision of the samples is primarily used in data compression techniques. 

The SPIHT~\cite{spiht1996} wavelet coefficient coding algorithm hierarchically partitions sets of
spatially related wavelet coefficients, exploiting the property that ``parent'' coefficients are
often larger in magnitude than ``child'' coefficients. This format allows regions to be
progressively refined in precision by coding more significant bitplanes before less significant
ones.

SBHP~\cite{sbhp2000}
JPEG2000~\cite{jpeg2001}
by bit plane streams~\cite{compression_techniques1991} (this is what SPIHT paper cites).
by magnitude streams~\cite{image_compression1992}
\peter{I'm not familiar with these last two papers.  Instead, point out that
SPIHT improves on embedded zerotree coding~\cite{ezw}.}

The other major research focus is precision reduction to compress the data. Quantization and
truncation.

\peter{Discuss scalar~\cite{sqe} vs. vector quantization~\cite{hvq}. SZ performs residual scalar
quantization~\cite{sz}. \cite{fpzip} truncates floats, which can be seen as nonuniform scalar
quantization.}

\newcommand{\zfp}{\textsc{zfp}\xspace}
Block transform-based techniques such as \zfp~\cite{zfp2014} partition the domain---a structured
grid---into small (e.g., $4 \times 4 \times 4$) independent blocks and thus allow for localized
decompression. Moreover, \zfp supports fixed-rate compression, which facilitates random access to
the data. Its fast transform and caching of decompressed data allows it to achieve not only high
throughput decompression~\cite{hvq}, but also fast inline compression. Extensions of \zfp allow it
to vary either the bit rate or precision spatially over the domain, albeit at fixed resolution.
\peter{Not sure if we want to cite my ARC poster on this: https://computation.llnl.gov/sites/default/files/public//llnl-post-728998.pdf}

\peter{Should the following go in a section on ``hybrid'' precision and resolution techniques?}

Woodring \etal~\cite{woodring2011} use the JPEG2000 image format to store floating point data.
Since most JPEG2000 implementations are limited to integer data, the authors apply uniform scalar
quantization to convert floating point data to integer form. Even though JPEG2000 supports varying
both resolution and precision, the authors do not explore this capability but focus only on setting
a bit rate.

\peter{Another wavelet paper relevant to VIS: \cite{treib}}.

Transfer function adaptive decompression~\cite{tf_decompression2004}

Transform Coding for HW-accel DVR~\cite{hw_dvr2007}

Tensor approximation for DVR~\cite{tensor_dvr2015}
\peter{This can be both resolution and precision.}

\peter{We might want to cite~\cite{codar} and~\cite{li} for surveys on data reduction.}

\section{Common terminologies}

This section defines common terminologies that are used throughout the paper.

\subsection{Pipeline}

Our goal is to associate each bit with a resolution level and a bit plane. To introduce the concept
of resolution to gridded data, we use the discrete wavelet transform, in particular, the popular
CDF5/3 [CITE] transform. For precision, we quantize the floating-point wavelet coefficients to
$16$-bit signed integers, thereby eliminating the exponent bits so that every bit can be associated
with a bit plane. To further avoid special treatment of the sign bit in two's complement form, we
convert the quantized coefficients to negabinary form [CITE]. This transformation increase the
number of bit planes from $16$ to $17$.

In practice, bits are never read and transmitted one by one, so in this paper, our fundamental unit
of data is a \emph{chunk} of bits. A \emph{chunks} is a bit plane of a group of $4\times 4$
coefficients in the same subband (the multi-dimensional wavelet transform rearranges the
coefficients into subbands [CITE], each can be thought of as a resolution level). Each group hence
contains $17$ chunks. For all the experiments in this paper, we perform the wavelet transform $3$
times in each dimension, producing $10$ subbands in 2D and $22$ subbands in 3D. Figure
\ref{fig:pipeline} illustrates how all these steps fit together to produce a stream of chunks from
the original, raw data.

\begin{figure}
  \centering
  \includegraphics[width=\linewidth]{img/pipeline.png}
  \caption{Our data stream creation pipeline. The input is a regular grid, the output is a stream of
  chunks, where each chunk is a bit plane from a group of quantized wavelet coefficients stored in
  negabinary format. The subbands are separated by blue lines in the second image, with the coarsest
  subband at the top left corner. TODO: revise this figure}
  \label{fig:pipeline}
\end{figure}

\subsection{Stream signature}

To analyze the core characteristics of a stream, we introduce the concept of a stream
\emph{signature}. A signature is a $B \times L$ matrix, with $B$ being the number of bit planes, and
$S$ being the number of subbands produced by the wavelet transform. A chunk on subband $s$ and bit
plane $b$ can be associated with the $b, l$ cell of the matrix. Every chunk in the stream can be
associated with one (and only one) of the cells. Each matrix cell $(b,l)$ contains a value in the
range of $[0,B\times L)$, indicating the position in which chunks of that cell appear in the stream,
relative to chunks of all other cells, on average. It is our hypothesis that streams optimized for
different analysis tasks produce qualitatively distinctive signatures, and therefore, a set of
signatures can be pre-computed once and later picked to use depending on the analysis at hand.

To compute a stream signature, we partition the whole domain into several regions, compute one
chunks in each per-region signature are spatially related. The reason for this partitioning is that
it is only meaningful to study the relative ordering of cells if the chunks associated with those
cells all come from same region of the original domain. Ideally the global signature for a stream
should contain all the per-region signatures, but due to space constraint, as well as for ease of
visualization, we condense all the local signatures by averaging them on a per-cell basis, and only
study the average signature. (TODO: add the algorithm).

% \begin{algorithm}
%   \KwData{this text}
%   \KwResult{how to write algorithm with \LaTeX2e }
%   initialization\;
%   \While{not at end of this document}{
%    read current\;
%    \eIf{understand}{
%     go to next section\;
%     current section becomes this one\;
%     }{
%     go back to the beginning of current section\;
%    }
%   }
%   \caption{How to write algorithms}
% \end{algorithm}

\section{Evaluation on Different Analysis Tasks}\label{sec:analysis-tasks}

\begin{table}[t]
	\caption{Data sets used in experiments. More data sets are included in the
	supplementary materials. In the main paper, every volume is of dimensions $64^3$. In the 
	supplementary materials, most volumes are of dimensions $256^3$.}
  \centering
  \begin{tabular}{p{0.08\textwidth}p{0.25\textwidth}p{0.12\textwidth}}
  \hline
  Name & Type & Data type \\
  \hline
  boiler & combustion simulation& float64\\
  plasma & magnetic reconnection simulation& float32\\
  diffusivity & hydrodynamics simulation& float64\\
  pressure & hydrodynamics simulation& float64\\
	turbulence & fluid dynamics simulation& float32\\
	kingsnake & CT scan & uint8\\
	foam & Scan? & uint16\\
	% flame & fluid dynamics simulation& float32 & x & x & x & x & x & Done\\
	% csafe & fluid dynamics simulation& uint8 & x &  &  & x & x & Done\\
	% enzo-v & fluid dynamics simulation& float32 & x & x &  & x &  & D(work2nd)\\
	% brain & fluid dynamics simulation& uint8 & x &  &  & x & x & Done\\
	% foam & fluid dynamics simulation& uint16 & x & x & x & x & x & Done\\
	% vismale & fluid dynamics simulation& uint8 & x & x & x & x & x & P(rerun)\\
	% karfs	& fluid dynamics simulation& float32 & x & x &  & x & x & Done\\
	% aneurism	& fluid dynamics simulation& uint8 & x & x &  & x & x & Done\\
  % velocityz & hydrodynamics simulation& float64 & x & x &  & x & x & D(main)\\
  \hline
  \end{tabular}\label{tbl:data-sets}
\end{table}

In this section, we consider a variety of common analysis tasks, namely function reconstruction
(\Cref{sec:rmse-optimized}), derivative computation (\Cref{sec:gradient,sec:laplacian}), histogram
computation (\Cref{sec:histogram}), and isocontour extraction (\Cref{sec:isocontour}). For each
task, we define an error metric $\err$ that is the basis for evaluating the performance of different
streams on the task. We use~\Cref{alg:greedy} to compute a stream $\sopt$ that is optimized for each
task, and use its signature to compute $\ssig$. We compare $\slvl$, $\sbit$, $\swav$, $\smag$,
$\ssig$, and $\sopt$ by evaluating the error as a function of the number of packets received. To
mimic the effects of entropy compression commonly used in practice, we remove all packets that
contain only leading zero bits from each stream before plotting. This comparison is performed on a
variety of data sets, listed in \Cref{tbl:data-sets}. The wavelet basis allows us to always
reconstruct data at full resolution, which greatly simplifies computations of errors (we are unaware
of standard methods to compute, for example, the root-mean-square error between grids of different
dimensions).

\subsection{Function Reconstruction}\label{sec:rmse-optimized}

\begin{figure*}[t]
\centering
 \subcaptionbox{\label{fig:rmse:boiler}\emph{boiler}}{{\includegraphics[width=0.24\linewidth]{rmse/rmse-optimized-boiler}}}
 \subcaptionbox{\label{fig:rmse:diffisivity}\emph{diffusivity}}{{\includegraphics[width=0.24\linewidth]{rmse/rmse-optimized-diffusivity}}}
 \subcaptionbox{\label{fig:rmse:plasma}\emph{plasma}}{ {\includegraphics[width=0.24\linewidth]{rmse/rmse-optimized-plasma}}}
 \subcaptionbox{\label{fig:rmse:turbulence}\emph{turbulence}}{{\includegraphics[width=0.24\linewidth]{rmse/rmse-optimized-turbulence}}}
\caption{Root-mean-square error (RMSE) of reconstructed functions for different streams and data
sets; lower RMSE is better. The streams are truncated to highlight the differences, without omitting
important information. Leading zero packets are not used for plotting. In all cases, the ordering of
error, from lowest to highest, is $\sopt < \ssig < \swav < \sbit < \smag < \slvl$.}
\label{fig:rmse-optimized}
%\vspace{1em}

\centering
 \subcaptionbox{\emph{by level} (\slvl)}{{\includegraphics[width=0.16\linewidth]{rmse/rmse-plasma-level}}}
 \subcaptionbox{\emph{by bit plane (\sbit)}}{{\includegraphics[width=0.16\linewidth]{rmse/rmse-plasma-bit-plane}}}
 \subcaptionbox{\emph{by wavelet norm (\swav)}}{{\includegraphics[width=0.16\linewidth]{rmse/rmse-plasma-wavelet-norm}}}
 \subcaptionbox{\emph{by magnitude (\smag)}}{{\includegraphics[width=0.16\linewidth]{rmse/rmse-plasma-magnitude}}}
 \subcaptionbox{\emph{by signature (\ssig)}}{{\includegraphics[width=0.16\linewidth]{rmse/rmse-plasma-signature}}}
 \subcaptionbox{\emph{ground truth}}{{\includegraphics[width=0.16\linewidth]{rmse/rmse-plasma-groundtruth}}}
\caption{Volume renderings of a $64^3$ region of \emph{plasma} data set at 0.1
bps.  \slvl captures the background (purple-blue) well, whereas \sbit captures
the fine details better. \swav combines the strength of both. \ssig, however,
produces the most accurate rendering (compare, e.g., yellow features).
\pavol{should we circle them or use an arrow?}\hb{yes, please}}
\label{fig:rmse-rendering}
\end{figure*}

One of the most fundamental analysis tasks is that of reconstructing the original function itself. A
commonly used error metric in this case is the root-mean-square error
(RMSE).~\Cref{fig:rmse-optimized} shows the comparison of the different streams for a variety of
datasets. It can be noted that, in general, $\sopt$ performs better than $\ssig$ due to spatial
adaptivity, while \ssig slightly outperforms \swav, followed by \sbit, \smag, and \slvl.

In particular, \sbit performs significantly better than \slvl, which can be attributed partly to our
removal of leading zero packets, because wavelet coefficients on finer scale subbands are much
smaller in magnitude. Such coefficients contain a majority of the leading zero bits, whose removal
benefits \sbit the most, as it touches fine-resolution bits the earliest. \smag underperforms for
the same reason that \slvl does, but to a lesser extent, since \smag is better adaptive to the data.
\swav outperforms both \slvl and \sbit, because it follows the optimal (data-independent) bit
ordering in $\strm_{L,B}$ in the $L_2$ norm, which is also the norm that RMSE is based upon.
Unsurprisingly, \sopt outperforms the others, as it is the most data-adaptive (i.e., it can
prioritize packets in the spatial domain in addition to the $\strm_{L,B}$ domain). \ssig is the
second best stream, as it follows the bit ordering of \sopt in $\strm_{L,B}$, but lacks spatial
adaptivity. In general, \swav and \ssig have similar performance. In some cases (e.g.,
~\Cref{fig:rmse:diffisivity} and~\Cref{fig:rmse:plasma}), \ssig is able to adapt more to the data,
thus outperforming \swav by a larger margin (\todo{explain}).

We explore the errors visually by volume rendering the \emph{plasma} data set at 0.1 bits per
sample, for all streams~(\Cref{fig:rmse-rendering}). Bits per sample (bps) are calculated by
dividing the total size of received packets (in bits) by the total number of samples. Although \slvl
has the precision to obtain an accurate background, it lacks resolution to resolve the fine details.
\sbit, instead, lacks the precision to reconstruct the (mostly smooth) background, but has enough
resolution to capture the fine details well. \swav balances both precision and resolution, producing
a more accurate picture as a whole. In this case, the \ssig stream manages to produce the most
accurate rendering. In general, \ssig benefits more from ``anisotropic'' data, such as
\emph{plasma}, where most features lie on a thin ``surface''. For such data, wavelet coefficients
along one dimension are often larger compared to those along other dimensions, which \sopt, and
hence \ssig, can take advantage of, but not \swav.

\subsection{Computation of derivatives}
\label{sec:derivatives}

Computation of derivative quantities such as gradient and Laplacian is of fundamental importance in
data analysis. In this section, we study streams that aim to minimize errors of derivative fields.
For the experiments in this section, we quantize the data to $32$ instead of $16$ bits, to ensure
enough precision for the purpose of taking derivatives using finite differences of floating-point
values. Note that in this paper, for the purpose of derivative computation, we always perform finite
differences on the finest (original) resolution. This is possible since the wavelet transform allows
for reconstruction of the function at the original resolution. The reason for this decision is to
avoid the problem of computing distances between quantities across grids of different dimensions
(e.g., computing the root-mean-square error between a (down-sampled) $n\times n$ grid and a
$2n\times 2n$ grid), because we are unaware of widely accepted solutions to this problem. In the
following sections, we perform experiments with two of the most common types of derivative, namely
gradient (Section~\ref{sec:gradient}) and Laplacian (Section~\ref{sec:laplacian}).

\subsubsection{Gradient}
\label{sec:gradient}

Since simulation data can rarely be captured by closed-form formulas, we use finite difference to
compute gradients. We experiment with three popular finite difference schemes using stencil size
widths of two, three, and five points in each dimension, but found no tangible differences in the
results. We have therefore decided to use the five-point stencil exlusively in this paper:
$\frac{\partial f}{\partial x}\approx
\frac{1}{12}f(x-2)-\frac{2}{3}f(x-1)+\frac{2}{3}f(x+1)-\frac{1}{12}f(x+2)$. In 2D, the gradient at
each grid point $(x,y)$ is the vector $(\frac{\partial f}{\partial x},\frac{\partial f}{\partial
y})$. We use Algorithm~\ref{alg:greedy} to compute a \emph{gradient-optimized} stream that minimizes
the difference between the gradient field of $f_b$ (the reconstructed function using $b$ bits per
sample) and that of the original function ($f$). At each grid point $p$, we compute an error $e(p)$,
defined to be the squared Euclidean length of the difference between two gradient vectors at $p$,
that is $e(p)=\norm{\nabla f_b(p)-\nabla f(p)}^2$. The overall error metric over the whole field,
$E_g$, is defined as $E_g(\nabla f_b,\nabla f)=\sqrt{\frac{1}{n}\sum_{i=1}^{n}{e(p_i)}}$.

We plot the gradient error curves produced by the \emph{by level}, \emph{by bit plane}, \emph{by
wavelet norm}, \emph{gradient-optimized}, and \emph{gradient signature} streams, for six data sets
(Figure~\ref{fig:gradient-error-comparison}). \emph{gradient-optimized} performs far better than all
the static streams, due to the fact that, with the knowledge of the data, this stream is able to
prioritize regions that would benefit the most from additional bits. Among the static streams,
\emph{by level} performs the worst, while \emph{by wavelet norm} performs slightly worse than the
other two streams. 

\begin{figure}[h]
	\centering
	\subcaptionbox{boiler}{
	{\includegraphics[width=0.48\linewidth]{gradient/gradient-optimized-boiler}}}
	\subcaptionbox{diffusivity}{
	{\includegraphics[width=0.48\linewidth]{gradient/gradient-optimized-diffusivity}}}
	\subcaptionbox{turbulence}{
	{\includegraphics[width=0.48\linewidth]{gradient/gradient-optimized-turbulence}}}
	\subcaptionbox{pressure}{
	{\includegraphics[width=0.48\linewidth]{gradient/gradient-optimized-pressure}}}
	\caption{Gradient error comparison among different streams, using the five-point stencil. The
	plots are truncated in the same way as in Figure \ref{fig:gradient-stencil-comparison}. In all
	cases, \emph{by wavelet norm} performs slightly worse than \emph{by bit plane} and \emph{gradient
	signature} do, but the differences are largely negligible. \emph{by level} performs the worst.}
	\label{fig:gradient-error-comparison}
\end{figure}

The differences in gradient errors among the \emph{by bit plane}, \emph{by wavelet norm}, and
\emph{gradient signature} streams are negligible. As an example, in
Figure~\ref{fig:gradient-rendering-diff} we visualize the three reconstructed gradient fields, as
well as the groundtruth gradient field for the \emph{velocityz} data set, at 0.35 bps, where the
streams diverge the most. This figure shows that the differences among reconstructed gradient fields
are barely visible. For this reason, in practice, any static stream would likely suffice for the
purpose of computing gradient. However, \emph{by bit plane} is significantly cheaper to compute
compared to \emph{gradient signature}, because the latter requires obtaining
\emph{gradient-optimized} first. \emph{by wavelet norm} is also a reasonable alternative, especially
if the bits are stored on disk in the same order, for example, to optimize for root-mean-square
errors (see Section~\ref{sec:motivation}. Finally, \emph{by wavelet norm} is also preferrable if,
beside an accurate gradient field, the task also requires an accurate function itself.

\begin{figure}[h]
	\centering
	\subcaptionbox{\emph{by level}}{
	{\includegraphics[width=0.31\linewidth]{gradient/gradient-turbulence-level}}}
	\subcaptionbox{\emph{by bit plane}}{
	{\includegraphics[width=0.31\linewidth]{gradient/gradient-turbulence-bit-plane}}}
	\subcaptionbox{\emph{by wavelet norm}}{
	{\includegraphics[width=0.31\linewidth]{gradient/gradient-turbulence-wavelet-norm}}}
	\subcaptionbox{\emph{by magnitude}}{
	{\includegraphics[width=0.31\linewidth]{gradient/gradient-turbulence-magnitude}}}
	\subcaptionbox{\emph{by signature}}{
	{\includegraphics[width=0.31\linewidth]{gradient/gradient-turbulence-signature.png}}}
	\subcaptionbox{\emph{groundtruth}}{
	{\includegraphics[width=0.31\linewidth]{gradient/gradient-turbulence-groundtruth.png}}}
	\caption{\emph{turbulence}, 0.2 bps}\label{fig:gradient-rendering-diff}
\end{figure}

Although \emph{gradient-optimized} and \emph{}

Figure~\ref{fig:gradient-renderings} gives a crude idea on the range of bit rates needed to
reconstruct a gradient field that is visually close to the ground-truth. For the data sets used in
our experiments, as low as 0.25 bps and as high as 2.39 bps are needed for this purpose.

\begin{figure}[h]
	\centering
	\subcaptionbox{$s_{bit}$}{
	{\includegraphics[width=0.48\linewidth]{gradient/gradient-bit-plane}}}
	\subcaptionbox{$s_{wav}$}{
	{\includegraphics[width=0.48\linewidth]{gradient/gradient-wavelet-norm}}}
	\caption{A 1D line extracted from \emph{plasma}, and reconstructed using $s_{bit}$ and $s_{wav}$ at
	0.6 bps. The original data is in orange, whille the reconstructions are in blue. $s_{wav}$ captures
	well the function values in low-gradient regions, where $s_{bit}$ struggles (red arrows).
	However, $s_{bit}$ retains the shape of the original function well in areas of both low and high
	gradients, where $s_{wav}$ instead produces smooth approximations (blue arrows). $s_{bit}$
	therefore is better for	derivative computations, where a function's shape (or its relative
	values), matter more than its absolute values.}\label{fig:gradient-rendering-diff}
\end{figure}

\subsubsection{Laplacian}\label{sec:laplacian}

The Laplace operator is a second-order differential operator, defined as the divergence of the
gradient field. It can be computed by summing second partial derivatives in all dimensions, for
example, in 2D: $\Delta f=(\frac{{\partial}^2}{\partial{x^2}}+\frac{{\partial}^2}{\partial{y^2}})f$.
To approximate the Laplacian for data on a grid, we use the three-point finite difference to
approximate the second derivative in each dimension: $\frac{{\partial}^2}{\partial{x^2}}f(x,y)
\approx f(x-1,y)-2f(x,y)+f(x+1,y)$. The Laplacian error is defined as the root-mean-square error
between the Laplacian of the reconstructed scalar field and the Laplacian of the original scalar
field, that is, $E_g(\Delta f_b,\Delta f)=RMSE(\Delta f_b,\Delta f)$. For each data set, Algorithm
[REF] is used to compute a \emph{laplacian-optimized} bit stream that minimizes $E_g$ at any bit
rate. From this stream, we obtain its signature and construct the \emph{laplacian signature} stream,
which, unlike \emph{laplacian-optimized}, is considered static and hence can be implemented in
practice. In Figure \ref{fig:laplacian-error-comparison} we compare these two stream against all the
previously defined streams, including \emph{gradient-optimized} using $E_g$ as the metric.

\begin{figure}[h]
	\centering
	\subcaptionbox{boiler}
	{\includegraphics[width=0.48\linewidth]{laplacian/laplacian-optimized-boiler}}
	\subcaptionbox{diffusivity}
	{\includegraphics[width=0.48\linewidth]{laplacian/laplacian-optimized-diffusivity}}
	\subcaptionbox{turbulence}
	{\includegraphics[width=0.48\linewidth]{laplacian/laplacian-optimized-turbulence}}
	\subcaptionbox{pressure}
	{\includegraphics[width=0.48\linewidth]{laplacian/laplacian-optimized-pressure}}
	\caption{Laplacian error comparison among streams, using the three-point stencil. The plots are
	truncated so as to better highlight differences without discarding important information. In all cases, \emph{laplacian}}
	\label{fig:laplacian-error-comparison}
\end{figure}

It can be observed that unlike the case for gradient, there exists significant differences between
the \emph{rmse-optimized} and \emph{laplacian-optimized} streams with regards. To understand these
differences we plot the precision of every wavelet coefficients at a low bit rate in Figure
\ref{fig:laplacian-precision-comparison} (a and b). When cross refererencing this Figure with Figure
\ref{fig:gradient-comparison}b we see that the \emph{laplacian-optimized} stream priotizes
finer-resolution bits where the sharp shockwave is, unlike the \emph{rmse-optimized} stream which
prefers lower-ordered, coarse-resolution bits. This effect makes sense intuitively, as the
derivative operator makes functions less smooth, hence amplifing hard edges. This happens in the
gradient case too, but to a much lesser degree.

\begin{figure}[h]
	\centering
	\subcaptionbox{\emph{by level}}
	{\includegraphics[width=0.31\linewidth]{laplacian/laplacian-pressure-level}}
	\subcaptionbox{\emph{by bit plane}}
	{\includegraphics[width=0.31\linewidth]{laplacian/laplacian-pressure-bit-plane}}
	\subcaptionbox{\emph{by magnitude}}
	{\includegraphics[width=0.31\linewidth]{laplacian/laplacian-pressure-magnitude}}
	\subcaptionbox{\emph{by wavelet norm}}
	{\includegraphics[width=0.31\linewidth]{laplacian/laplacian-pressure-wavelet-norm}}
	\subcaptionbox{\emph{by signature}}
	{\includegraphics[width=0.31\linewidth]{laplacian/laplacian-pressure-signature}}
	\subcaptionbox{\emph{groundtruth}}
	{\includegraphics[width=0.31\linewidth]{laplacian/laplacian-pressure-groundtruth}}
	\caption{pressure, laplacian, 0.9 bps}
	\label{fig:laplacian-precision-comparison}
\end{figure}

\emph{rmse-optimized}, \emph{laplacian-optimized}, and also \emph{gradient-optimized} for the euler
data set are visualized in Figure \ref{fig:signature-comparison}. 

TODO: add comparison of stream signatures
% \begin{figure}[h]
% 	\centering
% 	\subcaptionbox{\emph{rmse-optimized}}
% 	{\includegraphics[width=0.32\linewidth]{img/gradient-laplacian/SIG-GREEDY-(rmse).png}}
% 	\subcaptionbox{\emph{laplacian-optimized}}
% 	{\includegraphics[width=0.32\linewidth]{img/gradient-laplacian/SIG-GREEDY-(laplacian).png}}
% 	\subcaptionbox{\emph{gradient-optimized}}
% 	{\includegraphics[width=0.32\linewidth]{img/gradient-laplacian/SIG-GREEDY-(gradient).png}}
% 	\caption{Stream signatures visualized through a linear-blue color map (brighter is higher
% 	priority). From left to right: higher-ordered to lower-ordered bit planes. From top to bottom:
% 	coarser to finer subbands. Note that the streams from which the signatures are extracted do not
% 	contain leading zero bits, which explains the very dark cells }
% 	\label{fig:signature-comparison}
% \end{figure}

Using the signature for \emph{laplacian-optimized}, we are able construct a data-independent stream
(in the sense that once the signature is computed and is given, the ordering of the bits follows the
the signature only). This stream, called \emph{laplacian signature}, performs at least as well as,
and often better, than \emph{rmse-optimized} for all data sets (see Figure
\ref{fig:laplacian-comparison}). The reason \emph{laplacian signature} does not always outperform
\emph{rmse-optimized}, and that there is still a gap between itself and \emph{laplacian-optimized}
is that the signature is computed essentially by `'averaging'' local signatures, a process that
lessen the effectiveness of the signature when the data is highly inhomogenous (e.g., the euler data
set with its sharp shockwaves). Nevertheless, even with one signature for the whole domain, we are
able to reconstruct more accurate Laplacian in all cases in experiment. In practice, the signature
is a tiny piece of meta information that can be pre-computed, stored, and transmitted before any
value bits to help `'steer'' the data stream, whenever Laplacian is the quantity of interest.

\subsection{Histogram computation}\label{sec:histogram}

\pavol{motivate query; expand with examples}
\duong{what does this mean?}

The computation of a histogram is one of the most common tasks in data analysis. A histogram
succinctly summarizes the distribution of sample values, and thus is useful as a cursory ``look''
into the data, and in guiding further analysis. For example, it can be used to guide the selection
of colors and opacities in transfer functions.

There are several metrics proposed in the literature to measure the distance between two
distributions. We have experimented with the most common ones, namely Kolmogorov-Smirnov [CITE],
Kullback-Leibler [CITE], Hellinger [CITE], Total variation [CITE], Chi-square [CITE], Bhattacharyya
[CITE], Earth Mover's Distance~\cite{emd1998}, $L_1$ norm, and
Intersection~\cite{histogram_intersection1991}. We have found that the relative ordering of streams'
performance do not vary across the distance metrics. Also, it is interesting to note that among the
$s_{opt}$'s streams, the one using the Earth Mover's Distance has the least RMSE. However, we choose
Intersection as the metric of choice, because it is fast to compute, and is the least sensitive to
slight changes in precision.

\pavol{explain results in the curve plots}
With the error metric defined, we compute the {\em histogram-optimized} and {\em histogram
signature} streams. The remaining streams are static and we use ones from previous sections. For
every stream we compute the histogram intersection error and plot the
results~(\Cref{fig:histogram-stream-comparison}). The main difference between histogram query and
the other queries is that the {\em by level} stream outperforms both {\em by bit plane} and {\em by
magnitude} streams. This outcome is caused by the nature of a histogram, where its computation is
primarily dependant on the precision of the samples, i.e., even a small error in a sample value may
cause it to change bin in the histogram. The {\em by level} stream benefits from loading lower
resolutions at full precision as it can resolve the bins well. Moreover, since the subsampling is
uniform, it does not bias the histogram and preserves its shape well.

\begin{figure}[h]
	\centering
	\subcaptionbox{\emph{boiler}}
	{\includegraphics[width=0.48\linewidth]{histogram/histogram-optimized-boiler}}
	\subcaptionbox{\emph{diffusivity}}
	{\includegraphics[width=0.48\linewidth]{histogram/histogram-optimized-diffusivity}}
	\subcaptionbox{\emph{plasma}}
	{\includegraphics[width=0.48\linewidth]{histogram/histogram-optimized-plasma}}
	\subcaptionbox{\emph{turbulence}}
	{\includegraphics[width=0.48\linewidth]{histogram/histogram-optimized-turbulence}}
	\caption{Histogram error comparison among four streams \emph{histogram-optimized},
	\emph{rmse-optimized}, \emph{by wavelet norm}, and \emph{histogram signature}, without the leading
	zero bits. The plots are truncated to make the differences large enough for visual inspection, and
	truncation points are chosen so that the best among the reconstructed histograms is visually the
	same as the groundtruth histogram. }
	\label{fig:histogram-stream-comparison}
\end{figure}

%First, compared to \emph{rmse-optimized}, the \emph{histogram-optimized} stream produces
%consistently better histograms. Second, between the two data-independent streams, \emph{histogram
%signature} outperforms \emph{by wavelet norm}. To better understand how the Earth mover's distance
%translates to visual differences, we visualize the histograms produced by the different streams, at
%low bit rates, in Figure \ref{fig:histogram-comparison-low-bit-rate}. These plots confirm our
%observations.
%\emph{histogram-optimized} and \emph{histogram signature}
%outperform \emph{rmse-optimized} and \emph{by wavelet norm} respectively. For example, Figure
%\ref{fig:histogram-comparison-low-bit-rate-slz} shows the four reconstructed histograms at 0.1 bits
%per sample, with leading zero bits removed, for the diffusivity data set.

\pavol{explain visual results}
Additionally to computing the error curves, we explored several bit rates after the curves exhibit
reasonable error and the histograms are not wildy different from the groundtruth. For example, the {\em boiler}
data set's histogram for each streams confirms the results from the error curves~(\Cref{fig:histograms-boiler}).
The {\em by bit plane} stream or the {\em by magnitude} stream has vastly different shape and scale than the other
streams. As mentioned before, the primary cause is that histograms require precision more than resolution, and since these streams\pavol{todo: by magnitude}
always load full resolution but at limited precision, they spent large portion of the bit budget on the resolution part.
Any fo the remaining streams provides good histogram and thus the choice from these streams should depend on the other
desired queries.

\begin{figure}[h]
	\centering
	\subcaptionbox{\emph{by level}}{
	{\includegraphics[width=0.31\linewidth]{histogram/histogram-boiler-level.png}}}
	\subcaptionbox{\emph{by bit plane}}{
	{\includegraphics[width=0.31\linewidth]{histogram/histogram-boiler-bit-plane.png}}}
	\subcaptionbox{\emph{by magnitude}}{
	{\includegraphics[width=0.31\linewidth]{histogram/histogram-boiler-magnitude.png}}}
	\subcaptionbox{\emph{by wavelet norm}}{
	{\includegraphics[width=0.31\linewidth]{histogram/histogram-boiler-wavelet-norm.png}}}
	\subcaptionbox{\emph{by signature}}{
	{\includegraphics[width=0.31\linewidth]{histogram/histogram-boiler-signature.png}}}
	\subcaptionbox{\emph{groundtruth}}{
	{\includegraphics[width=0.31\linewidth]{histogram/histogram-boiler-groundtruth.png}}}
	\caption{Histograms of different streams of the \emph{boiler} data set at 0.08 bps. The
        {\em by level}, {\em by wavelet norm}, and {\em by signature}~\pavol{elsewhere histogram signature} streams
        produce histograms that have similar shape to the {\em groundtruth} histogram with most of the peaks and valleys
        preserved. In contrast, even though the {\em by bit plane} stream has similar mass distribution, it has one spurious
        peak that is not present in the {\em groundtruth} histogram.
        \pavol{are all those histograms normalized? why is there no greedy histogram?}}
	\label{fig:histograms-boiler}
\end{figure}

\pavol{summarize subsection}
In summary, we have evaluated six data sets, for each using different histogram streams, and
compared results with respect to the histogram intersection error and the visual differences. Interestingly, the
order of streams differs from all the other queries, where {\em by level} stream performed poorly, but for
histograms it outperforms both {\em by magnitude} and {\em by bit plane} streams.
The idea of signature works for histograms as well, and could be employed as a practical streaming
format, where the sender would first send the signature to the receiver at small cost (few integer
values). However, {\em by wavelet norm} stream has similar performance and does not require any extra
data transfered (such as signature), and thus we conclude that from
considered streams it has best properties.

% \begin{figure}
% 	\centering
% 	\subcaptionbox{\emph{histogram-optimized}}
% 	{\includegraphics[width=0.48\linewidth]{img/histogram/histogram-signature.png}}
% 	\subcaptionbox{\emph{rmse-optimized}}
% 	{\includegraphics[width=0.48\linewidth]{img/histogram/histogram-by-wavelet-norm.png}}
% 	\caption{euler's histograms at 0.18 bps. The groundtruth histogram is rendered in red, while the
% 	reconstructed histogram is rendered in green. The dark yellow regions are where the two overlap.}
% 	\label{fig:histogram-comparison-low-bit-rate-slz}
% \end{figure}


\pavol{The visualization of the subbands is missing, but I think it was quite interesting and maybe we
should find a way to include it as it makes the discussion more interesting.}
%The main difference between the \emph{histogram-optimized} and the \emph{rmse-optimized} streams is
%that \emph{histogram-optimized} favors low-ordered bits of coarse-level coefficients, while,
%\emph{rmse-optimized} relatively favors high-ordered bits of fine-level coefficients. This is
%evident in Figure \ref{fig:histogram-signature-comparison} (b and d): for
%\emph{histogram-optimized}, the bright blue cells extend more toward the right (lower-ordered bit
%planes) and less toward the bottom (finer resolution levels). The histogram experiments in this
%Section are performed with 256 bins, but this fact holds for a wide range of number of bins. Figures
%\ref{fig:histogram-signature-comparison} (a, b, c) show that varying the number of bins from 128 to
%512 only affects the relative ordering among the low-ordered bits on very fine resolution levels
%(the dark blue cells at the bottom right of the signature). These are bits that come at the end of a
%stream, and thus matter little to the data quality. 

%The streams in Figure \ref{fig:histogram-comparison-low-bit-rate} (b, d, f, h) are truncated where
%the EMD errors of the \emph{histogram-optimized} streams are negligible, suggesting that it is often
%possible to achieve near lossless histograms with just 1 bit per sample. The boiler data set is a
%peculiar case, where the \emph{histogram-optimized} stream outperforms the rest of the streams by a
%large margin in the second half of the bit rate range. Looking at the precision maps (defined in
%Section [REF]) for the \emph{histogram-optimized} stream, as well as its reconstructed histogram
%(Figure \ref{fig:precision-map-histogram}, (a)), we see that the bit distribution is heavily
%concentrated in regions corresponding to one particular histogram bin that contains vastly more
%samples than other bins do. This situtation happens when there are many samples having essentially
%the same value but they are distributed irregularly in space (otherwise they would form constant
%regions which would be captured very well with just a few precision bits of wavelet coefficients --
%this is the case for the flame data set). In this case a stream needs to be more spatially adaptive
%to resolve well the histogram bin with the most samples, and among the tested streams, only
%\emph{histogram-optimized} is spatially adaptive to EMD (\emph{rmse-optimized} is spatially
%adaptive, but to RMSE, and the other two streams are data-independent).

% \begin{figure}[h]
% 	\centering
% 	\subcaptionbox{}
% 	{\includegraphics[width=0.24\linewidth]{img/histogram/boiler/prec-histogram_resize-vert.png}}
% 	\subcaptionbox{}
% 	{\includegraphics[width=0.24\linewidth]{img/histogram/boiler/prec-rmse_resize-vert.png}}
% 	\subcaptionbox{}
% 	{\includegraphics[width=0.24\linewidth]{img/histogram/boiler/prec-signature_resize-vert.png}}
% 	\subcaptionbox{}
% 	{\includegraphics[width=0.24\linewidth]{img/histogram/boiler/prec-wavenorm_resize-vert.png}}
% 	\caption{(top) Precision distribution of wavelet coefficients, and (bottom) reconstructed
% 	histograms, for \emph{histogram-optimized}, \emph{rmse-optimized}, \emph{by wavelet norm}, and
% 	\emph{histogram signature} streams, at 6.47 bits per sample, without leading zero bits.}
% 	\label{fig:precision-map-histogram}
% \end{figure}

\duong{Discuss the case of not using compression, where the signature plays a more important role.}

%In this section we have shown that histogram computation requires a different ordering of bits than
%reconstructing the function itself. We have also proposed a practical heuristic, based on stream
%signatures, to capture the main characteristics of this ordering. In practice, the histogram
%signature can be pre-computed once and stored on disk. A signature's size is negligible (170
%integers in 2D and 374 integers in 3D in the case of four wavelet levels in each dimension), and can
%also be further compressed. Therefore it can be transmitted first, and the receiver of the data can
%ultilize the signature to smartly query the bits so as to reconstruct the data's histogram with as
%few bits as possible.
%%% Local Variables:
%%% mode: latex
%%% TeX-master: "template"
%%% End:


\section{Isocontour extraction}
If histograms provide statistical overviews of the range space of data, isocontour (isosurface)
extraction reveals data's structures and features of interest, such as objects or organs in medical
imaging, or a flame's front in combustion simulation. Sequences of isocontours can reveal important
topological structures in data. Extraction of isocontour is therefore an essential task in any
visualization and analysis system. In this section we study the characteristics of bit streams that
minimize errors in the reconstructed contour at all time.

As before, we begin by defining an error metric two compare two contours. A popular metric is the
Hausdorff distance [CITE] between two curves (in 2D) or two surfaces (in 3D), but we have found that
using the number of misclassified samples result is a more robust mertic. This metric can be
computed without actually extracting contours. Given two grid of values $G_1$ and $C_2$, and an
isovalue $v$, we classify each sample of each grid with a binary label, $0$ means the sample's value
is less than $v$, and $1$ means the opposite. Then, we perform an element-wise XOR of the resulting
two binary grids, and count the number of $1$'s in the result. This number is the number of samples
that were classified differently between the two grids, and in our experience, is a close proxy to
how different the two corresponding isocontours would be from each other.

Using just the number of misclassified voxels as the error metric in Algorithm [REF] is
insufficient, however. We have found that in step TODO of the algorithm, if the error caused by
switching off a chunk is too small (in orders of sub-pixel/sub-voxel), then the importance of the
chunk cannot be properly measured. We therefore amend the error metric by adding to it the relative
difference in length between two contours. This relative difference is, most of the time, a number
between $0$ and $1$,computed by the formula $|L(C_1)-L(C_2)|/L(C_1)$, where $L(C)$ is the length of
a contour $C$. The idea is that when the number number of misclassified voxels is less than $1$, the
error -- which is at the sub-pixel level -- is captured by the relative difference in contour length
instead.

With an error metric defined, we can compute an \emph{isocontour-optimized} stream for a data set,
given an isovalue. However, it is not straightforward to compare an \emph{isocontour-optimized}
stream to an \emph{rmse-optimized} stream in any contour error metric, because in practice, an
isocontour occupies an order of magnitude fewer grid cells, compared to the whole grid. An
\emph{rmse-optimized} stream has no reason to prioritize these cells, which is what an
\emph{isocontour-optimized} stream would do. In other words, the \emph{isocontour-optimized} stream
is spatially adaptive to a much higher degree than the \emph{rmse-optimized} stream is, and this
makes a direct comparison less meaningful. One way to solve this problem is to isolate the spatial
adaptivity aspect of a stream from its core characteristics (i.e., its tendency to favor precision
or resolution). Our solution is to divide space into regions, and study the relative order in which
a stream consumes bits within a region (in a way similar to how we construct a stream's signature),
while keeping the ordering of regions constant across streams.

To do so, we construct \emph{hybrid} streams that combines two given streams (the relative order
matters). For example, the \emph{hybrid isocontour-rmse} combines \emph{isocontour-optimized} and
\emph{rmse-optimized} in a way such that within a region, it has the same chunk ordering as that of
\emph{rmse-optimized}'s, but globally, it visits the regions in the same order that
\emph{isocontour-optimized} does. Another hybrid stream that can be formed is \emph{hybrid
rmse-isocontour}, which is form similarly to how \emph{hybrid isocontour-rmse} is formed, but with
its two input streams switching roles. Since a hybrid stream is designed to retain the
characteristics of the first stream locally (within a region), it can be used as a proxy of the
second stream for comparison against the first. The \emph{hybrid isocontour-rmse} stream can be
compared with \emph{isocontour-optimized}, and the \emph{hybrid rmse-isocontour} stream can be
compared with \emph{rmse-optimized}. Figure \ref{fig:isocontour-plots} provides these comparisons.

\begin{figure}
	\centering
	\subcaptionbox{boiler}
	{\includegraphics[width=0.48\linewidth]{img/isocontour/boiler-isocontour.pdf}}
	\subcaptionbox{boiler}
	{\includegraphics[width=0.48\linewidth]{img/isocontour/boiler-isocontour-long.pdf}}
	\subcaptionbox{kingsnake}
	{\includegraphics[width=0.48\linewidth]{img/isocontour/kingsnake-isocontour.pdf}}
	\subcaptionbox{kingsnake}
	{\includegraphics[width=0.48\linewidth]{img/isocontour/kingsnake-isocontour-long.pdf}}
	\subcaptionbox{pressure}
	{\includegraphics[width=0.48\linewidth]{img/isocontour/pressure-isocontour.pdf}}
	\subcaptionbox{viscosity}
	{\includegraphics[width=0.48\linewidth]{img/isocontour/pressure-isocontour-long.pdf}}
	\subcaptionbox{viscosity}
	{\includegraphics[width=0.48\linewidth]{img/isocontour/viscosity-isocontour.pdf}}
	\subcaptionbox{pressure}
	{\includegraphics[width=0.48\linewidth]{img/isocontour/viscosity-isocontour-long.pdf}}
	\caption{Isocontour errors for \emph{rmse-optimized}, \emph{isocontour-optimized}, and their
	hybrid streams. The vertical axis is log-scaled. The bit rates are capped to highlight differences
	among streams.}
	\label{fig:isocontour-plots}
\end{figure}

The first thing to observe from Figure \ref{fig:isocontour-plots} is that the \emph{hybrid
isocontour-rmse} stream improves on \emph{rmse-optimized}, as expected. However, it underperforms
\emph{isocontour-optimized} by a considerable degree. This could be due to the fact that
\emph{hybrid isocontour-rmse} is more spatially adaptive to the contour, or that \emph{hybrid
isocontour-rmse} is fundamentally different than \emph{rmse-optimized}. To answer this question, we
form a special stream 

figure 3: show isocontour rendering for the three streams at some low bit rates where the errors are
apparent. Ideally the difference between isocontour and hybrid is not noticeable.

We argue that different isovalues will require different stream signatures, but thanks to the
similarity between hybrid and isocontour streams, the rmse stream can be used to extract isocontour
too, provided that it is paired with a method to localize the contour (e.g. min-max octrees etc).

Figure 5: We show that the hybrid and isocontour can diverge somewhat for low-gradient contour.
Valerio suggested here we coudl also build a "ramp" dataset at different angles and see if the two
diverges more as the ramp become flatter.

We argue that if the gradient is low, some noise bits at the end will make an impact, the isocontour
is very sensitive to noise, and is in general not interesting or meaninfgul to extract.

\section{Discussion and Future Work}

{\color{blue}This work addresses one of the biggest challenges in scientific visualization today,
which is the enormous amount of data being generated. We focus on an issue rarely discussed in the
literature, which is the trade-off between two prominent dimensions of data reduction, namely
resolution and precision, when performing analysis tasks. To keep the study tractable while not
compromising the generalizability of results, we avoid multi-parameter tasks such as volume
rendering, in hopes that the fundamental analysis tasks considered here can serve as building blocks
for more complex tasks. Although we are more concerned with the quality of underlying data rather
than visualization outputs, we believe the two often strongly correlate, as demonstrated throughout
the paper. We have also chosen error metrics to measure data quality so that they can be good
proxies for evaluating visualization outputs. The paper focuses on a small set of core tasks, but
the framework is generic and applies to any well-defined metric, and one future direction is to
consider a broader set of tasks.}

We present the first empirical study to demonstrate that combining reduction in precision and
resolution can lead to a significant improvement in data quality for the same data size, and that
different tasks might prefer different resolution-versus-precision trade-offs. For example, while
computing histograms requires high precision, computing derivatives benefits more from higher
resolution, and function reconstruction and isosurface extraction require a proper mix of the two
(see~\Cref{fig:bit-distrib}). We also show that common reduction techniques, e.g., those based on
\slvl and \smag, do not perform well when leading zero bits are removed (to simulate entropy
compression). {\color{red}For each task, the relative ordering of the rate-distortion curves stay
largely the same regardless of data sets, although the gaps between them vary depending on the
smoothness and noisiness of the data. Compared to data-independent streams, signature-based streams
often perform better because they are more adaptive to the data. They are also amenable for
implementation (unlike \sopt), as a signature is negligibly small and thus can be precomputed and
stored during pre-processing. It is also interesting to consider per-block signatures instead of a
global one.}

{\color{red}An important question is whether task- and data-dependent streams provide sufficient
advantages over purely data-independent streams. In practice, data would be used for multiple,
not-necessarily predefined tasks and maintaining multiple streams will likely lead to additional
overheads. Here, we consider \ssig to be the best possible stream that could be realized.
Improvements on \ssig in the resolution-versus-precision space are likely possible, but unlikely to
be significant. Given these assumptions and the fact that \ssig in most cases provides very similar
results to \sbit or \swav, the additional effort (and potential overheads) for task-dependent bit
orderings is unlikely to be beneficial. This leaves a significant gap between the best
data-independent streams and the optimal stream \sopt. Our experiments suggest that the majority of
this difference can be attributed to spatial adaptivity. The prototypical example are isosurfaces
where \sopt can skip all regions that do not effect any portion of the surface. It is worthwhile in
future work to investigate solutions to spatial adaptivity to significantly improve the performance
of data-independent streams.}

This study can be considered only a first step towards a system of solutions that can optimize
storage, network, and I/O bandwidth to suit specific tasks at hand. Ultimately, our results can
guide development of new data layouts and file formats for scientific data. It seems \swav provides
the best all around performance. However, for a file format additional constraints must be
considered, such as disk block sizes, cache coherence, and compression. Furthermore, an ideal format
should allow task-dependent data queries even though task-dependent formats will likely be
restricted to very specific situations. 

Another important consideration is that for fair comparisons we always reconstruct the data at full
resolution using wavelets. However, in practice, processing and memory costs are important, and it
is likely that adaptive representations would be used~\cite{gigavoxels,Gobbetti2008,vdb2013}. In
these cases the error of a given approximation depends not only on the information that is
available, but also on what algorithm and what interpolation method are used. For example,
trilinear interpolation on a coarse grid might provide vastly different results than wavelet
reconstruction on the original mesh. It is possible to specifically construct grids where both
interpolations are equivalent~\cite{weiss}, yet these are not yet implement in standard tools. An
important future research direction will be to understand the implications of the results presented
here on existing toolchains such as VTK.

%%% Local Variables:
%%% mode: latex
%%% TeX-master: "template"
%%% End:


%% if specified like this the section will be committed in review mode
% \acknowledgments{
% The authors wish to thank A, B, and C. This work was supported in part by
% a grant from XYZ (\# 12345-67890).}

\clearpage
\clearpage

%\bibliographystyle{abbrv}
\bibliographystyle{abbrv-doi}
%\bibliographystyle{abbrv-doi-narrow}
%\bibliographystyle{abbrv-doi-hyperref}
%\bibliographystyle{abbrv-doi-hyperref-narrow}

\bibliography{egbibsample}
\end{document}

