\subsection{Histogram computation}\label{sec:histogram}

\pavol{motivate query; expand with examples}
\duong{what does this mean?}

The computation of a histogram is one of the most common tasks in data analysis. A histogram
succinctly summarizes the distribution of sample values, and thus is useful as a cursory ``look''
into the data, and in guiding further analysis. For example, it can be used to guide the selection
of colors and opacities in transfer functions.

There are several metrics proposed in the literature to measure the distance between two
distributions. We have experimented with the most common ones, namely Kolmogorov-Smirnov [CITE],
Kullback-Leibler [CITE], Hellinger [CITE], Total variation [CITE], Chi-square [CITE], Bhattacharyya
[CITE], Earth Mover's Distance~\cite{emd1998}, $L_1$ norm, and
Intersection~\cite{histogram_intersection1991}. We have found that the relative ordering of streams'
performance do not vary across the distance metrics. Also, it is interesting to note that among the
$s_{opt}$'s streams, the one using the Earth Mover's Distance has the least RMSE. However, we choose
Intersection as the metric of choice, because it is fast to compute, and is the least sensitive to
slight changes in precision.

\pavol{explain results in the curve plots}
With the error metric defined, we compute the {\em histogram-optimized} and {\em histogram
signature} streams. The remaining streams are static and we use ones from previous sections. For
every stream we compute the histogram intersection error and plot the
results~(\Cref{fig:histogram-stream-comparison}). The main difference between histogram query and
the other queries is that the {\em by level} stream outperforms both {\em by bit plane} and {\em by
magnitude} streams. This outcome is caused by the nature of a histogram, where its computation is
primarily dependant on the precision of the samples, i.e., even a small error in a sample value may
cause it to change bin in the histogram. The {\em by level} stream benefits from loading lower
resolutions at full precision as it can resolve the bins well. Moreover, since the subsampling is
uniform, it does not bias the histogram and preserves its shape well.

\begin{figure}[h]
	\centering
	\subcaptionbox{\emph{boiler}}
	{\includegraphics[width=0.48\linewidth]{histogram/histogram-optimized-boiler}}
	\subcaptionbox{\emph{diffusivity}}
	{\includegraphics[width=0.48\linewidth]{histogram/histogram-optimized-diffusivity}}
	\subcaptionbox{\emph{plasma}}
	{\includegraphics[width=0.48\linewidth]{histogram/histogram-optimized-plasma}}
	\subcaptionbox{\emph{turbulence}}
	{\includegraphics[width=0.48\linewidth]{histogram/histogram-optimized-turbulence}}
	\caption{Histogram error comparison among four streams \emph{histogram-optimized},
	\emph{rmse-optimized}, \emph{by wavelet norm}, and \emph{histogram signature}, without the leading
	zero bits. The plots are truncated to make the differences large enough for visual inspection, and
	truncation points are chosen so that the best among the reconstructed histograms is visually the
	same as the groundtruth histogram. }
	\label{fig:histogram-stream-comparison}
\end{figure}

%First, compared to \emph{rmse-optimized}, the \emph{histogram-optimized} stream produces
%consistently better histograms. Second, between the two data-independent streams, \emph{histogram
%signature} outperforms \emph{by wavelet norm}. To better understand how the Earth mover's distance
%translates to visual differences, we visualize the histograms produced by the different streams, at
%low bit rates, in Figure \ref{fig:histogram-comparison-low-bit-rate}. These plots confirm our
%observations.
%\emph{histogram-optimized} and \emph{histogram signature}
%outperform \emph{rmse-optimized} and \emph{by wavelet norm} respectively. For example, Figure
%\ref{fig:histogram-comparison-low-bit-rate-slz} shows the four reconstructed histograms at 0.1 bits
%per sample, with leading zero bits removed, for the diffusivity data set.

\pavol{explain visual results}
Additionally to computing the error curves, we explored several bit rates after the curves exhibit
reasonable error and the histograms are not wildy different from the groundtruth. For example, the {\em boiler}
data set's histogram for each streams confirms the results from the error curves~(\Cref{fig:histograms-boiler}).
The {\em by bit plane} stream or the {\em by magnitude} stream has vastly different shape and scale than the other
streams. As mentioned before, the primary cause is that histograms require precision more than resolution, and since these streams\pavol{todo: by magnitude}
always load full resolution but at limited precision, they spent large portion of the bit budget on the resolution part.
Any fo the remaining streams provides good histogram and thus the choice from these streams should depend on the other
desired queries.

\begin{figure}[h]
	\centering
	\subcaptionbox{\emph{by level}}{
	{\includegraphics[width=0.31\linewidth]{histogram/histogram-boiler-level.png}}}
	\subcaptionbox{\emph{by bit plane}}{
	{\includegraphics[width=0.31\linewidth]{histogram/histogram-boiler-bit-plane.png}}}
	\subcaptionbox{\emph{by magnitude}}{
	{\includegraphics[width=0.31\linewidth]{histogram/histogram-boiler-magnitude.png}}}
	\subcaptionbox{\emph{by wavelet norm}}{
	{\includegraphics[width=0.31\linewidth]{histogram/histogram-boiler-wavelet-norm.png}}}
	\subcaptionbox{\emph{by signature}}{
	{\includegraphics[width=0.31\linewidth]{histogram/histogram-boiler-signature.png}}}
	\subcaptionbox{\emph{groundtruth}}{
	{\includegraphics[width=0.31\linewidth]{histogram/histogram-boiler-groundtruth.png}}}
	\caption{Histograms of different streams of the \emph{boiler} data set at 0.08 bps. The
        {\em by level}, {\em by wavelet norm}, and {\em by signature}~\pavol{elsewhere histogram signature} streams
        produce histograms that have similar shape to the {\em groundtruth} histogram with most of the peaks and valleys
        preserved. In contrast, even though the {\em by bit plane} stream has similar mass distribution, it has one spurious
        peak that is not present in the {\em groundtruth} histogram.
        \pavol{are all those histograms normalized? why is there no greedy histogram?}}
	\label{fig:histograms-boiler}
\end{figure}

\pavol{summarize subsection}
In summary, we have evaluated six data sets, for each using different histogram streams, and
compared results with respect to the histogram intersection error and the visual differences. Interestingly, the
order of streams differs from all the other queries, where {\em by level} stream performed poorly, but for
histograms it outperforms both {\em by magnitude} and {\em by bit plane} streams.
The idea of signature works for histograms as well, and could be employed as a practical streaming
format, where the sender would first send the signature to the receiver at small cost (few integer
values). However, {\em by wavelet norm} stream has similar performance and does not require any extra
data transfered (such as signature), and thus we conclude that from
considered streams it has best properties.

% \begin{figure}
% 	\centering
% 	\subcaptionbox{\emph{histogram-optimized}}
% 	{\includegraphics[width=0.48\linewidth]{img/histogram/histogram-signature.png}}
% 	\subcaptionbox{\emph{rmse-optimized}}
% 	{\includegraphics[width=0.48\linewidth]{img/histogram/histogram-by-wavelet-norm.png}}
% 	\caption{euler's histograms at 0.18 bps. The groundtruth histogram is rendered in red, while the
% 	reconstructed histogram is rendered in green. The dark yellow regions are where the two overlap.}
% 	\label{fig:histogram-comparison-low-bit-rate-slz}
% \end{figure}


\pavol{The visualization of the subbands is missing, but I think it was quite interesting and maybe we
should find a way to include it as it makes the discussion more interesting.}
%The main difference between the \emph{histogram-optimized} and the \emph{rmse-optimized} streams is
%that \emph{histogram-optimized} favors low-ordered bits of coarse-level coefficients, while,
%\emph{rmse-optimized} relatively favors high-ordered bits of fine-level coefficients. This is
%evident in Figure \ref{fig:histogram-signature-comparison} (b and d): for
%\emph{histogram-optimized}, the bright blue cells extend more toward the right (lower-ordered bit
%planes) and less toward the bottom (finer resolution levels). The histogram experiments in this
%Section are performed with 256 bins, but this fact holds for a wide range of number of bins. Figures
%\ref{fig:histogram-signature-comparison} (a, b, c) show that varying the number of bins from 128 to
%512 only affects the relative ordering among the low-ordered bits on very fine resolution levels
%(the dark blue cells at the bottom right of the signature). These are bits that come at the end of a
%stream, and thus matter little to the data quality. 

%The streams in Figure \ref{fig:histogram-comparison-low-bit-rate} (b, d, f, h) are truncated where
%the EMD errors of the \emph{histogram-optimized} streams are negligible, suggesting that it is often
%possible to achieve near lossless histograms with just 1 bit per sample. The boiler data set is a
%peculiar case, where the \emph{histogram-optimized} stream outperforms the rest of the streams by a
%large margin in the second half of the bit rate range. Looking at the precision maps (defined in
%Section [REF]) for the \emph{histogram-optimized} stream, as well as its reconstructed histogram
%(Figure \ref{fig:precision-map-histogram}, (a)), we see that the bit distribution is heavily
%concentrated in regions corresponding to one particular histogram bin that contains vastly more
%samples than other bins do. This situtation happens when there are many samples having essentially
%the same value but they are distributed irregularly in space (otherwise they would form constant
%regions which would be captured very well with just a few precision bits of wavelet coefficients --
%this is the case for the flame data set). In this case a stream needs to be more spatially adaptive
%to resolve well the histogram bin with the most samples, and among the tested streams, only
%\emph{histogram-optimized} is spatially adaptive to EMD (\emph{rmse-optimized} is spatially
%adaptive, but to RMSE, and the other two streams are data-independent).

% \begin{figure}[h]
% 	\centering
% 	\subcaptionbox{}
% 	{\includegraphics[width=0.24\linewidth]{img/histogram/boiler/prec-histogram_resize-vert.png}}
% 	\subcaptionbox{}
% 	{\includegraphics[width=0.24\linewidth]{img/histogram/boiler/prec-rmse_resize-vert.png}}
% 	\subcaptionbox{}
% 	{\includegraphics[width=0.24\linewidth]{img/histogram/boiler/prec-signature_resize-vert.png}}
% 	\subcaptionbox{}
% 	{\includegraphics[width=0.24\linewidth]{img/histogram/boiler/prec-wavenorm_resize-vert.png}}
% 	\caption{(top) Precision distribution of wavelet coefficients, and (bottom) reconstructed
% 	histograms, for \emph{histogram-optimized}, \emph{rmse-optimized}, \emph{by wavelet norm}, and
% 	\emph{histogram signature} streams, at 6.47 bits per sample, without leading zero bits.}
% 	\label{fig:precision-map-histogram}
% \end{figure}

\duong{Discuss the case of not using compression, where the signature plays a more important role.}

%In this section we have shown that histogram computation requires a different ordering of bits than
%reconstructing the function itself. We have also proposed a practical heuristic, based on stream
%signatures, to capture the main characteristics of this ordering. In practice, the histogram
%signature can be pre-computed once and stored on disk. A signature's size is negligible (170
%integers in 2D and 374 integers in 3D in the case of four wavelet levels in each dimension), and can
%also be further compressed. Therefore it can be transmitted first, and the receiver of the data can
%ultilize the signature to smartly query the bits so as to reconstruct the data's histogram with as
%few bits as possible.
%%% Local Variables:
%%% mode: latex
%%% TeX-master: "template"
%%% End:
