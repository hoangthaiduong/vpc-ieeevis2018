\subsection{Histogram computation}\label{sec:histogram}

The computation of a histogram is one of the most common tasks in data analysis. A histogram
succinctly summarizes the distribution of sample values, and thus is useful as a cursory ``look''
into the data and in guiding further analysis. For example, it can be used to guide the selection
of colors and opacities in transfer functions.

There are several metrics proposed in the literature to measure the distance between two
distributions. We have experimented with the most common ones, namely Kolmogorov-Smirnov [CITE],
Kullback-Leibler [CITE], Hellinger [CITE], Total variation [CITE], Chi-square [CITE], Bhattacharyya
[CITE], Earth Mover's Distance~\cite{emd1998}, $L_1$ norm, and
Intersection~\cite{histogram_intersection1991}. We have found that the relative ordering of streams'
performance does not vary across the distance metrics. Also, it is interesting to note that among
the $s_{opt}$'s streams, the one using the Earth Mover's Distance has the least RMSE\pavol{maybe add
in the data so it is clear it is not on the histogram bins?}. However, we choose Intersection as the
metric of choice, because it is fast to compute and is the least sensitive to slight changes in
precision. The intersection distance between two histograms $H_1$ and $H_2$ is defined as
$e(H_1,H_2)=\sum_{i}{\min{(H_1(i),H_2(i))}}$ (the sum is over all bins $i$). Every histogram is
normalized by dividing each bin by the total number of samples in the volume. Our error metric takes
into account not only the shapes but also the value ranges of the histograms. Therefore, when computing
the histogram of a reconstructed function, we clamp its range of values to that of the original
function, so that corresponding histogram bins, i.e., $H_1(i)$ and $H_2(i)$, share the same range.
TODO: add that the histogram intersection is insensitive to the number of bins

As before, for each data set, we use~\Cref{alg:greedy} to compute an $s_{hist-opt}$ stream,
optimized for histogram error, and an $s_{hist-sig}$ from its signature. We plot the error curves
for all relevant streams using the error metric just
defined~(compare \Cref{fig:histogram-stream-comparison}). We use 64 for the number of bins, but note that
there exist no meaningful differences across a wide range of number of bins (from 64 to 512) in our
experiments. In all cases, the group consisting of $s_{bit}$, $s_{lvl}$, and $s_{mag}$ underperforms
the other group by a large margin.

Among the former group, $s_{lvl}$ generally outperforms $s_{bit}$ at low bit rates, although there
are several crossover points between the two curves. These crossover points are explained
in~\Cref{fig:histogram-explain}. When leading zero packets are present, $s_{lvl}$
outperforms $s_{bit}$, because increasing resolution does
not help producing an accurate histogram as much as increasing precision. The histogram is oblivious
to spatial locations of samples (which requires resolution to resolve), but it is sensitive to
sample values (which requires precision). However, when leading zero packets are removed, as is the
case when using compression, $s_{bit}$ benefits significantly more than $s_{lvl}$ does (for the same
reason explained in~\Cref{sec:rmse-optimized}), resulting in the observed crossovers. Finally,
$s_{mag}$ performs poorly, because it ignores regions of smooth variations, which nevertheless count
toward the distribution.

In the other group, the performances of $s_{wav}$ and $s_{hist-sig}$ (and even $s_{hist-opt}$)
differ by negligible amount. This observation is confirmed in~\Cref{fig:histograms-boiler}\pavol{Fig. as the figure
labels are like that), where we
plot various histograms, reconstructed at 0.08 bps, for the \emph{boiler} data set. The histograms
produced by $s_{wav}$ and $s_{hist-sig}$ have approximately the same shape, and are the closest to
the reference histogram. The next best histogram is produced by $s_{lvl}$, followed by the one
produced by $s_{bit}$, and finally $s_{mag}$. These results suggest that histogram computation is
among the analysis tasks that benefit significantly from a bit ordering that combines both resolution
and precision, not one that adheres to either exclusively.

\begin{figure}[h]
	\centering
	\subcaptionbox{\emph{boiler}}
	{\includegraphics[width=0.48\linewidth]{histogram/histogram-optimized-boiler}}
	\subcaptionbox{\emph{diffusivity}}
	{\includegraphics[width=0.48\linewidth]{histogram/histogram-optimized-diffusivity}}
	\subcaptionbox{\emph{plasma}}
	{\includegraphics[width=0.48\linewidth]{histogram/histogram-optimized-plasma}}
	\subcaptionbox{\emph{turbulence}}
	{\includegraphics[width=0.48\linewidth]{histogram/histogram-optimized-turbulence}}
	\caption{Comparison of histogram errors among streams. Plots are truncated to highlight
	differences without hiding important trends. In general, $s_{hist-opt}\approx s_{hist-sig}\approx
	s_{wav} > s_{lvl} > s_{bit} > s_{mag}$. Crossover points between $s_{bit}$ and $s_{lvl}$ are
	explained in~\Cref{fig:histogram-explain}\pavol{figure referencing figure is ehm}}.
	\label{fig:histogram-stream-comparison}
\end{figure}

\begin{figure}[h]
	\centering
	\subcaptionbox{with leading zero packets}
	{\includegraphics[width=0.48\linewidth]{histogram/histogram-explain-boiler-wlz}}
	\subcaptionbox{without leading zero packets}
	{\includegraphics[width=0.48\linewidth]{histogram/histogram-explain-boiler}} \caption{Comparison
	of histogram error curves produced by $s_{bit}$ and $s_{lvl}$, for \emph{boiler}, with and without
	leading zero bits. The vertical axis is in log scale. The error for $s_{bit}$ reduces in a
	stair-step fashion, in which each step corresponds to a new bit plane streamed. $s_{bit}$ benefits
	significantly more from the removal of leading zero bits (from (a) to (b), the blue curve shifts
	more to the left).}
	\label{fig:histogram-explain}
\end{figure}

\begin{figure}[h]
	\centering
	\subcaptionbox{\emph{by level} ($s_{lvl}$)}{
	{\includegraphics[width=0.31\linewidth]{histogram/histogram-boiler-level.png}}}
	\subcaptionbox{\emph{by bit plane} ($s_{bit}$)}{
	{\includegraphics[width=0.31\linewidth]{histogram/histogram-boiler-bit-plane.png}}}
	\subcaptionbox{\emph{by magnitude} ($s_{mag}$)}{
	{\includegraphics[width=0.31\linewidth]{histogram/histogram-boiler-magnitude.png}}}
	\subcaptionbox{\emph{by wavelet norm} ($s_{wav}$)}{
	{\includegraphics[width=0.31\linewidth]{histogram/histogram-boiler-wavelet-norm.png}}}
	\subcaptionbox{\emph{by signature} ($s_{hist-sig}$)}{
	{\includegraphics[width=0.31\linewidth]{histogram/histogram-boiler-signature.png}}}
	\subcaptionbox{\emph{reference}}{
	{\includegraphics[width=0.31\linewidth]{histogram/histogram-boiler-groundtruth.png}}}
	\caption{Histograms of the \emph{boiler} data set, reconstructed at 0.08 bps. $s_{lvl}$,
	$s_{wav}$, and $s_{hist-sig}$ produce histograms that share a shape similar to the reference histogram,
	with most of the peaks and valleys preserved. In contrast, $s_{bit}$ produces a spurious peak not
	found in the reference. Finally, $s_{mag}$'s histogram has a widely skewed distribution where too
	many values fall into the first bin.}
	\label{fig:histograms-boiler}
\end{figure}
