As the gap between the available compute power and the cost of data movement
increases, data transfer, whether from cache, main memory, or from disk,
becomes a major bottleneck in many workflows. However, it has been shown that
not every bit of data is always necessary to answer scientific questions with
required accuracy. In particular, for techniques at the end of scientific
workflows, such as visualization and data analysis, lower fidelity
representations of the data often provide adequate
approximations~\cite{woodring2011,covra2012,compression_sim2013}, and even
during simulation some loss in precision is often
acceptable~\cite{compression_sim2013,doi:10.1177/1094342018762036}. As a
result, a several different techniques have been proposed to reduce the size
of data. 

Broadly, these techniques can be classified into (i) reducing the data
resolution, e.g., the number of data points stored, and (ii) reducing the
precision of each data point.  Examples of the former kind of approaches are subsampling~\cite{Pascucci01sc},
adaptive mesh refinement~\cite{amr1989}, octrees, or other tree
structures~\cite{hierarchical1984}, \peter{What about subsampling, as in IDX?
And what about wavelets and other multires structures?}, and those of the
latter are compression~\cite{fpzip,isabela,zfp2014,sz} or
quantization~\cite{vq1992,compression_domain2003,sqe}. Traditionally,
multiresolution structures have been used to accelerate dynamic queries, e.g.,
in rendering~\cite{multires_octree1999}, since discarding data points based on
the viewpoint or data complexity can result in significant speed-up.
Compression based on uniform quantization, on the other hand, is more common
when storing data, where in the absence of other information, treating each
sample as equally important is the null hypothesis. However, in many
situations, a combination of both resolution and precision reduction could be
appropriate.  For example, high spatial resolution may be needed to resolve the
topology of an isosurface, yet the corresponding data samples may be usable at
less than full precision to adequately approximate the geometry.  Conversely,
accumulating accurate statistics may require high-precision values, yet
a lower resolution subset of points may be sufficient for the task. 

In general, different
\hbadd{levels of adaptivity in}\hbdel{combinations of} 
resolution and precision may be suitable for different types of analysis and visualization tasks,
and for many, these requirements will be
data dependent. Consequently, a globally optimal data organization may not exist. 
Instead, we consider a progressive setting in which some initial data is loaded
and processed, and new data is requested selectively based on the requirement
of the current task as well as the characteristics of the data already loaded. The
result is a stream of bits ordered such that it minimizes the error, considering the task at hand.
However, although intuitively there are almost certainly advantages in considering both resolution and precision in the ordering it is unclear how much the error could be reduced for a given data budget or how little data could be used to achieve the same error.
Furthermore, optimal data dependent orderings especially may not be practical since they assume perfect knowledge of the data. It is therefore important to understand which of these potential gains are realizable.
This paper aims to answer these important questions through extensive, empirical experiments. 
In particular, our contributions are:

\begin{itemize}
%
\item A framework that allows systematic studies of the
resolution-versus-precision tradeoffs for fundamental data analysis and
visualization tasks. The core idea is to represent various data reduction
techniques as data streams that improve data quality in either resolution or
precision in each step (\Cref{sec:terminologies}). We can
thus compare these techniques fairly, by comparing the corresponding data
streams.
%  
\item Empirical evidence that jointly optimizing resolution and
precision can provide significant improvements on the results of analysis tasks
over adjusting either independently.  This claim is demonstrated using a
collection of data sets and data analysis tasks. We also show how different
types of data analysis might require substantially different data streams for
optimal results.
%
\item Lower bounds of error \peter{You either estimate the error or
provide a bound---not both.} \ptb{I am not sure I agree we have lower bounds on the error} for various analysis tasks using a greedy approach
that jointly optimizes resolution and precision (\Cref{sec:data_dep_streams}).
\peter{Too much going on here.  An error bound, a greedy algorithm, joint
``optimization'' of precision and resolution---whatever that might mean.  How
are these all related?} In addition, we also identify practical streams that
closely approximate these bounds for each task (\Cref{sec:rmse-optimized},
\Cref{sec:derivatives}, \Cref{sec:histogram}, and \Cref{sec:isocontour}), using
a novel concept called \emph{stream signature} (\Cref{sec:stream-signature}).
\peter{Add one sentence that gives a concise description of what a ``stream
signature'' is.}
\end{itemize}


%%% Local Variables:
%%% mode: latex
%%% TeX-master: "template"
%%% End:
