As the gap between the available compute power and the cost of data movement increases, data
transfer, whether from cache, main memory, or disk, is becoming the major bottleneck of many
workflows. However, it is well known that often not every bit of data is necessary to answer
scientific questions at required accuracy. In particular, for techniques at the end of scientific
workflows, such as visualization and data analysis, much coarser approximations of the data often
provide adequate approximations~\cite{woodring2011,covra2012,compression_sim2013}, and even during
simulation some loss in precision is often acceptable~\cite{compression_sim2013}. As a result, a
host of different techniques have been proposed to reduce the size of data. Generally, these
techniques can be classified into approaches that reduce the data resolution, i.e., the number of
data points stored, and techniques that reduce the precision of each data point.

Examples of the former are adaptive mesh refinement~\cite{amr1989}, octrees or other tree
structures~\cite{hierarchical1984}, and of the latter compression~\cite{zfp2014} or
quantization~\cite{vq1992}. Traditionally, multi-resolution structures have been used to accelerate
dynamic queries, for example, in rendering~\cite{multires_octree1999}, since discarding data points
based on the view point or data complexity can result in significant speed-ups. Compression based on
uniform quantization on the other hand is more common when storing data, where in the absence of
other information, treating each sample as equally important is the null-hypothesis. However, in
many situations a combination of both resolution and precision reduction could be appropriate. For
example, we may need a high spatial resolution to resolve the topology of an isosurface, yet it is unlikely we
need the corresponding data samples at full precision to adequately approximate the
geometry. Conversely, we may need high precision values to accumulate accurate statistics, yet a
lower-resolution subset of points may be sufficient to do so. 

In general, there exists a wide range of potential analysis and visualization tasks all of which may
prefer different combinations of resolution and precision, and for many, these requirements will be
data-dependent. Consequently, we do not expect any particular data organization to be ideal for all
cases. Instead, we consider a progressive setting in which some initial data is loaded and processed,
and new data is requested selectively based on the requirement of the current task and the
characteristics of the data already loaded. The result is a stream of bits that should minimize
the error of the task at hand. However, while intuitively there are almost certainly advantages in
adapting both resolution and precision, it is unclear how big the potential gains could be and how
much of the benefits can be realized with practical algorithms. This paper aims to answer these
important questions through extensive, empirical experiments. Our contributions are:

\begin{itemize}
\item We present a framework that allows systematic studies of the resolution-versus-precision
  trade-offs for fundamental data analysis and visualization tasks. The core idea is to represent
  various data reduction techniques as data streams that improve data quality in either resolution
  or precision in each step (\autoref{sec:terminologies}). We can thus compare these techniques
  fairly, by comparing the corresponding data streams.
   
\item We provide empirical evidence on how jointly optimizing resolution and precision can provide
  significant improvements on the results of analysis tasks over adjusting either independently.
  This claim is demonstrated using a collection of data sets and data analysis tasks. We also show
  how different types of data analysis might require substantially different data streams for
  optimal results.

\item We estimate lower bounds of error for various analysis tasks using a greedy aprroach that
  jointly optimizes resolution and precision (\autoref{sec:data_dep_streams}). In addition, we also
  identify practical streams that closely approximate these bounds for each task
  (\autoref{sec:rmse-optimized}, \autoref{sec:derivatives}, \autoref{sec:histogram}, and
  \autoref{sec:isocontour}), using a novel concept called \emph{stream signature}
  (\autoref{sec:stream-signature}).
\end{itemize}

%%% Local Variables:
%%% mode: latex
%%% TeX-master: "EGauthorGuidelines-eurovis18-full"
%%% End:
