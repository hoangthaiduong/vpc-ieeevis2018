\section{Introduction}

As the gap between the available compute power and the cost of data
movement increases, data transfer, whether from caches, main memory,
or disk, is becoming the major bottleneck of many workflows. However,
it is well known that often not every byte of data is necessary to
answer scientific questions at sufficient accuracy.
In particular, for techniques at the end of scientific
workflow, like visualization and data analysis, much coarser
approximations of the data often provide adequate approximations~\cite{woodring2011,covra2012,compression_sim2013},
               and even during simulation some loss in precision is often
               acceptable~\cite{compression_sim2013}. As a result, a host of different
               techniques have been proposed to reduce the size of data. Generically,
               these can be classified into approaches that reduce the data
               resolution, i.e., the number of data points or coefficients stored, and
               techniques that reduce the precision of each data point or coefficient.
               Examples of the former are
               adaptive mesh refinement~\cite{amr1989}, octrees or other tree
               structures~\cite{hierarchical1984} and of the latter compression~\cite{zfp2014},
               quantization~\cite{vq1992}, or truncation~\cite{?}.
               \note{What does `truncation' refer to?  Discarding mantissa bits?}
               Traditionally,
               multi-resolution structures have been used to accelerate dynamic
               queries, for example, in rendering~\cite{multires_octree1999}, since discarding data points
               based on the view point or data complexity can result in significant
               speed-ups. Compression based on uniform quantization on the other hand is
               more common when storing data, where in the absence of other
               information treating each sample as equally important is the
               null-hypothesis. However, in many situations a combination of both
               resolution and precision reduction would be appropriate.
               For example, we may need a certain minimal
               spatial resolution to adequately resolve the geometry of an isosurface,
               yet if the underlying function is smooth we likely will not need all values
               at full precision.
               \note{I see no link between smoothness and precision.  Derivatives require
                       high precision; visualization doesn't.}
                       Similarly, we may need high precision values to
                       accumulate accurate statistics, yet a small subsample of points may be
                       sufficient to do so.
                       \note{Are we making this up, or do we have an example where such high
                               precision is needed?}
                               Some multi-resolution file formats like JPEG2000~\cite{jpeg2001} do
provide per-block progressive decompression and thus allow at least some
flexibility
in adjusting both resolution and precision simultaneously. But we are
not aware of any analysis pipeline that uses these capabilities beyond
simple region-of-interest based queries. In fact, while intuitive, it
is unclear whether combining both concepts provides any
advantage in practice and how big the potential savings could, as well as
how flexible a given solution would have to be.\note{What does this flexibility
refer to?}

\note{This makes sense only in a progressive analysis pipeline.  We need
to make that clear, as well as the notion of both progressive resolution
and precision.}
Using a wide range of data sets and a suite of different progressive analysis
tasks, we show that selectively adjusting both resolution and precision can
provide significant advantages over adjusting only one of them. In
particular, given an analysis task, we consider progressive
bitstreams that improve the ordering of the bits (or small chunks of bits)
of all samples to reduce the error of the analysis. This ordering includes
interleaving bits of different samples and resolutions across the entire
data set. We show that reordered streams can not only provide---sometimes
drastically---improved results, but that different types of
analysis require substantially different bitstreams. This result implies that
there exists no universally good ordering to replace the
multi-resolution or precision based streams used today.
\note{Point out earlier that standard techniques order by resolution of
by bit plane only.  I don't think that's clear.}
Instead, an ideal system should support adaptive and dynamic change in the
bit-ordering dependening on the task.
\note{And dependent on the data?}
While creating such a system is
beyond the scope of this paper, we demonstrate heuristics that provide
good approximations to the theoretical ideal, which is unachievable in
practice. Our contributions is detail are:

\begin{itemize}

\item Using a suite of data sets and analysis tasks, we demonstrate
  that jointly optimizing both resolution and precision provides
  significant data reductions above and beyond  adjusting each
  separately;

\item An optimization algorithm approximates the ideal
  bitstream for large data and thus enables extensive, empirical
  experiments; and
\note{``Enabling extensive, empirical experiments'' is a contribution?
Clarify why this is a contribution.}

\item A practical heuristic to dynamically create
  high-quality bitstreams for different analysis tasks.

\end{itemize}

 



% In recent years, compute bandwidth has been greatly outpacing the
% latency of data movement across all levels of the memory hierarchy,
% making data movement the bottleneck in almost all workflows that
% involve data creation or processing. When it comes to analyzing the
% terabytes of data produced by massively parallel simulation, this
% bottleneck prevents scientists from studying the data to gain
% scientific insights. Several data reduction techniques have been
% proposed to reduce the amount of data for storage and analysis, and
% they mainly fall into two categories: techniques that reduce data's
% resolution (e.g., octrees, wavelets), and techniques that reduce
% data's precision (e.g., quantization and truncation). As we will show
% in this paper, however, reducing data only in resolution or only in
% precision is not flexible enough. Much larger gain can be achieved by
% combining these two dimensions of data reductions.

% There exists data reduction techniques that combine resolution and precision, of which perhaps the
% most well known is JPEG2000. The specifications for JPEG2000 however do not handle high-precision 3D
% data commonly found in scientific applications. But more importantly, JPEG2000 data streams do not
% adapt to the analysis task at hand. We will show in this paper that a data reduction scheme should
% be task-dependent to maximize the reducibility potential, as different analysis tasks have different
% preferences in terms of the resolution and precision of data. These preferences possibly also change
% across the whole domain. It is therefore important to understand the characteristics of the analysis
% tasks to avoid reading and transmitting unimportant data bits.

% For a wide range of data sets, we construct different bit streams, each optimized for a common
% analysis task, and study their characteristics. The study presented in this paper is a step toward a
% data representation that allows bits pertaining to a given analysis task to be read efficiently,
% with minimal amount of unnecessary bits. Such a data representation would enable interactive
% exploration as well as cursory analysis of enormous, possibly remote data sets, without waiting for
% days of data transmission before figuring out what regions of the data to focus on.

% Our contributions are, in detail:

% \begin{itemize}
%   \item For fundamental analysis tasks, namely data reconstruction, gradient and Laplacian computation, iso-contour extraction, and histogram computation, we introduce a framework in which spatially adaptive data streams in both resolution and precision can be formed and compared. We also devise an optimization algorithm that approximates the optimal bit stream for each task. These streams serve as both empirical limits and sources of insights to designing heuristics to task-dependent .
%   \item Based on observations on the aforementioned task-optimized streams, we propose practical heuristics for:
%   \begin{itemize}
%     \item Reconstruction the function data accurately in the L2 norm, as well as its first and second derivatives (gradient and Laplacian).
%     \item Extraction of an iso-contour given the iso-value, by localizing the contour spatially, and applying the heuristic for minimizing the L2 norm of function error locally.
%     \item Computation of histograms, using a novel piece of ``metadata'' information called \emph{stream signature}.
%   \end{itemize}
% \end{itemize}


%%% Local Variables:
%%% mode: latex
%%% TeX-master: "EGauthorGuidelines-eurovis18-full"
%%% End:
