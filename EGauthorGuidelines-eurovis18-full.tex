% ---------------------------------------------------------------------------
% Author guideline and sample document for EG publication using LaTeX2e input
% D.Fellner, v1.13, Jul 31, 2008

\documentclass{egpubl}
\usepackage{eurovis2018}
\usepackage{subcaption}
\Full_EuroVis
% --- for  Annual CONFERENCE
% \ConferenceSubmission   % uncomment for Conference submission
%\ConferencePaper        % uncomment for (final) Conference Paper
% \STAR                   % uncomment for STAR contribution
% \Tutorial               % uncomment for Tutorial contribution
% \ShortPresentation      % uncomment for (final) Short Conference Presentation
% \Areas                  % uncomment for Areas contribution
% \MedicalPrize           % uncomment for Medical Prize contribution
% \Education              % uncomment for Education contribution
%
% --- for  CGF Journal
% \JournalSubmission    % uncomment for submission to Computer Graphics Forum
% \JournalPaper         % uncomment for final version of Journal Paper
%
% --- for  CGF Journal: special issue
% \SpecialIssueSubmission    % uncomment for submission to Computer Graphics Forum, special issue
\SpecialIssuePaper         % uncomment for final version of Journal Paper, special issue
%
% --- for  EG Workshop Proceedings
% \WsSubmission    % uncomment for submission to EG Workshop
% \WsPaper         % uncomment for final version of EG Workshop contribution
%
 \electronicVersion % can be used both for the printed and electronic version

% !! *please* don't change anything above
% !! unless you REALLY know what you are doing
% ------------------------------------------------------------------------

% for including postscript figures
% mind: package option 'draft' will replace PS figure by a filname within a frame
\ifpdf \usepackage[pdftex]{graphicx} \pdfcompresslevel=9
\else \usepackage[dvips]{graphicx} \fi

\PrintedOrElectronic

% prepare for electronic version of your document
\usepackage{t1enc,dfadobe}

\usepackage{egweblnk}
\usepackage{cite}
\usepackage{cleveref}
\usepackage{amsmath}


% For backwards compatibility to old LaTeX type font selection.
% Uncomment if your document adheres to LaTeX2e recommendations.
% \let\rm=\rmfamily    \let\sf=\sffamily    \let\tt=\ttfamily
% \let\it=\itshape     \let\sl=\slshape     \let\sc=\scshape
% \let\bf=\bfseries

% end of prologue

\title[EG \LaTeX\ Author Guidelines]%
{Progressive data streams}

% for anonymous conference submission please enter your SUBMISSION ID
% instead of the author's name (and leave the affiliation blank) !!
\author[D. Fellner \& S. Behnke]
{D.\,W. Fellner\thanks{Chairman Eurographics Publications Board}$^{1,2}$
  and S. Behnke$^{2}$
  %        S. Spencer$^2$\thanks{Chairman Siggraph Publications Board}
  \\
  % For Computer Graphics Forum: Please use the abbreviation of your first name.
  $^1$TU Darmstadt \& Fraunhofer IGD, Germany\\
  $^2$Institut f{\"u}r ComputerGraphik \& Wissensvisualisierung, TU Graz, Austria
  %        $^2$ Another Department to illustrate the use in papers from authors
  %             with different affiliations
}

% ------------------------------------------------------------------------

% if the Editors-in-Chief have given you the data, you may uncomment
% the following five lines and insert it here
%
% \volume{27}   % the volume in which the issue will be published;
% \issue{1}     % the issue number of the publication
% \pStartPage{1}      % set starting page


%-------------------------------------------------------------------------
\begin{document}
  
% \teaser{
%  \includegraphics[width=\linewidth]{eg_new}
%  \centering
%   \caption{New EG Logo}
% \label{fig:teaser}
% }

\maketitle

\newcommand{\norm}[1]{\left\lVert#1\right\rVert}
\begin{abstract}
Performing scientific analysis on the petabytes of data commonly found nowadays is an onerous task
due to the prohibitive cost of data transfer. Currently this issue is alleviated by working with
reduced-resolution or reduced-precision data. This paper brings together techniques that reduce data
in resolution, precision, and both, into a common framework in which they can be studied and
compared. Their error patterns are compared with regards to fundamental metrics, namely L2 norm,
first and second derivatives, histograms, and isocontours. We also compute and study
metric-dependent, data-optimal streams which serve as empirical bounds on the limit to which
reduction can happen, given an error tolerance for the analysis task at hand. Finally, based on the
observed characteristics of those streams, we propose practical heuristics to minimize the amount of
data needed to perform a given analysis task. The insights and heuristics presented here can be
leveraged to implement data-optimal representations and querying systems to facilitate interactive
exploration as well as cursory analysis of enormous data.  
\begin{classification} % according to http://www.acm.org/class/1998/
  \CCScat{Computer Graphics}{I.3.3}{Picture/Image Generation}{Line and curve generation}
\end{classification}
  
\end{abstract}

%\input{EGauthorGuidelines-body.inc}
\section{Introduction}

Please follow the steps outlined in this document very carefully when
submitting your manuscript to Eurographics.

You may as well use the \LaTeX\ source as a template to typeset your own
paper. In this case we encourage you to also read the \LaTeX\ comments
embedded in the document.


The list of our contributions:

\begin{itemize}
        \item We show that traditional multi-resolution data streaming schemes that group bits either in the same coefficient or in the same bit plane result in suboptimal errors. We reduce error significantly using the same number of bits by allowing bits belonging to the same bit plane or the same coefficient to be streamed separately.

        \item We devise a greedy method to approximate an optimal bit ordering for several quantities of interest, namely RMSE, Histogram, Gradient, Laplacian, Isocontour, Volume rendering. Based on these orderings, we observe that there are more than one optimal ordering for Gradient/Laplacian/Histogram, in particular, the RMSE ordering is also nearly optimal for these quantities.

        \item We show that locally the optimal bit orderings for all tested quantities are just slight variations of the optimal RMSE ordering. Furthermore, locally the (data-dependent) RMSE ordering is approximately the same as a data-independent ordering that can be computed using only the norms of the wavelet basis functions.

        \item We describe a practical algorithm to realize the aforementioned data-independent optimal ordering, taking advantage of an encoding scheme that compress the leading zero bits. We also note that this ordering is globally (as opposed to locally) near optimal in terms of RMSE. 
\end{itemize}

\section{Related work}

Techniques that reduce data in resolution typically build a tree-shape hierarchy over the data. A
very common scheme to genearte such a hierarchy is to construct low-resolution copies of the data
from higher-resolution ones through filtering and subsampling. Examples include Gaussian and
Laplacian pyramids~\cite{laplacian-pyramid},
mipmaps~\cite{multires_octree1999,interactive-exploration-ct-scans}. Often times, the data is stored
in blocks on each resolution level. To save bandwidth, low-resolution blocks can be streamed during
rendering, if the points being queried project to less than a pixel on screen. However, these methods
increase storage requirements, making them unsuitable for very large data.

Recent multi-resolution techniques save storage by adapting to the data in such a way that different
regions of the data are stored in different resolutions, depending on how ``homogeneous'' the region
is. Fast-varying regions are stored at higher resolution. A very popular approach is sparse voxel
octrees (SVO), pioneered by Crassin \etal~\cite{gigavoxels} and Gobbetti \etal~\cite{Gobbetti2008},
and variations of which are found in~\cite{Fogal-2013-RayGuided,visualization-driven}. Sparsity
comes from the fact that smooth-varying regions are stored at coarser octree levels, which
significantly reduces storage. During rendering, blocks of samples are streamed from an appropriate
resolution, determined by how far the queried samples are from the eye/camera. Beyer
\etal~\cite{large‐scale-volume} gives a comprehensive overview of this family of techniques.

Other trees such as B+ tree~\cite{vdb2013} and kd-tree~\cite{fogal-kdtree} can also be used in lieu
of octrees to build a sparse hierarchy. Alternatively, space-filling curves such as the hierarhical
Z curve~\cite{idx2001} can be used to reorder data samples to form a hierarchy without any filtering
steps or redundant samples, as low-resolution levels are constructed via subsampling. Unfortunately,
subsampling can introduce heavy aliasing artifacts.

\duong{continue from here}
A sparse multiresolution hierarchy offers level-of-detail access, which reduces bandwidth. 

wavelets~\cite{treib,multires_toolkit2003,vapor2007,woodring2011}.

For example, the Compression-domain Output-sensitive Volume Rendering Architecture
(COVRA)~\cite{covra2012} constructs a level-of-detail pyramid in a precomputation stage to reduce
memory usage for blocks that are far from the viewpoint. Moreover, COVRA further compresses the
blocks of the pyramid to reduce the data transfer time, either to the GPU or over a network.
However, this redundant level-of-detail representation increases data size, which may be undesirable
for some tasks.

wavelets:
~\cite{treib}
~\cite{multires-framework}
~\cite{compression-domain-volume-rendering}
~\cite{interactive-rendering-large-volume}
~\cite{rapid-compression-volume}
~\cite{survey-multires}
~\cite{multires_toolkit2003}
~\cite{wavelet-compression-interactive-vis}

resolution
~\cite{multires-volume-rendering} (wavelet, level of detail)
~\cite{in-situ-sampling-particle}

tensor:
~\cite{tensor_dvr2015}
~\cite{multiscale-tensor}
~\cite{tamresh}

Octrees:
~\cite{sph-octree} (octree + wavelet compression)

Other trees:

Error analysis:
~\cite{evaluating-compression-climate}
~\cite{compression_sim2013}
~\cite{statistical-volume-quality}
~\cite{evaluating-efficacy-wavelet}
~\cite{topology-verification-isosurface}
~\cite{verifiable-isosurface}
~\cite{verifying-volume-rendering}
~\cite{statistical-volume-quality}

vector quantization:
~\cite{vq1992}
~\cite{hw_dvr2007} (VQ on transfrom domain)
~\cite{compression_domain2003} (VQ after Haar-like transform)
~\cite{covra2012} octrees of bricks with sparse representations

error-guided: 
~\cite{tf_decompression2004} (based on transfer function)

~\cite{spgrid2014}

surveys:
~\cite{state-of-the-art-compressed-volume} (compressed volume rendering)
~\cite{li2018} data reduction techniques

compression:
~\cite{isabela}
~\cite{fpzip}
~\cite{sz}
~\cite{zfp2014}

precision:
~\cite{ezw}
~\cite{spiht1996}
~\cite{mloc}
~\cite{sbhp2000}
~\cite{jpeg2001}

The wavelet transform constructs a hierarchy of resolution levels via low and high bandpass filters.
The transform is recursively applied to the lower-resolution band, resulting in a hierarchy of
``details'' at varying resolution. One benefit of wavelets over redundant representations like
Laplacian pyramids is that the wavelet transform is merely a change of basis that like reordering
techniques does not increase the data size. Another benefit over AMR and other tree-like techniques
is that the wavelet basis functions are defined everywhere in space, requiring no special
interpolation rules when given some arbitrary subset of wavelet coefficients and basis functions.
One disadvantage of the wavelet transform is non-constant time random access cost, though
acceleration structures have been proposed to speed up local queries~\cite{weiss}.

Spatial adaptivity in resolution can be achieved by tiling the wavelet coefficients of individual
subbands. For example, the Visualization and Analysis Platform for Ocean, Atmosphere, and Solar
Researchers (VAPOR) toolkit~\cite{multires_toolkit2003, vapor2007} incorporates a multiresolution
file format based on a wavelets to allow data analysis on commodity hardware, and stores individual
tiles in separate files to allow loading of the region of interest. However, the authors only
leverage the resolution control without exploring the precision axis, which can potentially further
reduce data transfer.

Reducing precision of the samples is primarily used in data compression techniques. 

The SPIHT~\cite{spiht1996} wavelet coefficient coding algorithm hierarchically partitions sets of
spatially related wavelet coefficients, exploiting the property that ``parent'' coefficients are
often larger in magnitude than ``child'' coefficients. This format allows regions to be
progressively refined in precision by coding more significant bitplanes before less significant
ones.

SBHP~\cite{sbhp2000}
JPEG2000~\cite{jpeg2001}
by bit plane streams~\cite{compression_techniques1991} (this is what SPIHT paper cites).
by magnitude streams~\cite{image_compression1992}
\peter{I'm not familiar with these last two papers.  Instead, point out that
SPIHT improves on embedded zerotree coding~\cite{ezw}.}

The other major research focus is precision reduction to compress the data. Quantization and
truncation.

\peter{Discuss scalar~\cite{sqe} vs. vector quantization~\cite{hvq}. SZ performs residual scalar
quantization~\cite{sz}. \cite{fpzip} truncates floats, which can be seen as nonuniform scalar
quantization.}

\newcommand{\zfp}{\textsc{zfp}\xspace}
Block transform-based techniques such as \zfp~\cite{zfp2014} partition the domain---a structured
grid---into small (e.g., $4 \times 4 \times 4$) independent blocks and thus allow for localized
decompression. Moreover, \zfp supports fixed-rate compression, which facilitates random access to
the data. Its fast transform and caching of decompressed data allows it to achieve not only high
throughput decompression~\cite{hvq}, but also fast inline compression. Extensions of \zfp allow it
to vary either the bit rate or precision spatially over the domain, albeit at fixed resolution.
\peter{Not sure if we want to cite my ARC poster on this: https://computation.llnl.gov/sites/default/files/public//llnl-post-728998.pdf}

\peter{Should the following go in a section on ``hybrid'' precision and resolution techniques?}

Woodring \etal~\cite{woodring2011} use the JPEG2000 image format to store floating point data.
Since most JPEG2000 implementations are limited to integer data, the authors apply uniform scalar
quantization to convert floating point data to integer form. Even though JPEG2000 supports varying
both resolution and precision, the authors do not explore this capability but focus only on setting
a bit rate.

\peter{Another wavelet paper relevant to VIS: \cite{treib}}.

Transfer function adaptive decompression~\cite{tf_decompression2004}

Transform Coding for HW-accel DVR~\cite{hw_dvr2007}

Tensor approximation for DVR~\cite{tensor_dvr2015}
\peter{This can be both resolution and precision.}

\peter{We might want to cite~\cite{codar} and~\cite{li} for surveys on data reduction.}

% \section{The advantages of combining resolution and precision}
% \label{sec:motivation}

% Traditional data reduction techniques work by truncating either the finest few resolution levels, or
% the last few low-ordered bits, but rarely both. In this section, we show that significant gain can
% be achieved through reducing both data resolution and precision. For each data set, we construct and
% compare four different streams, namely \emph{by level}, \emph{by bit plane}, \emph{by wavelet norm},
% and \emph{by magnitude}. The first three streams were introduced in
% Section~\ref{sec:common-static-streams}. The \emph{by magnitude} stream mimics a common data
% reduction scheme in the literature, in which the smaller-magnitude wavelet coefficients are removed.
% Correspondingly, in \emph{by magnitude}, the chunks are sorted by the sum of magnitude of their
% respective coefficients, and chunks of the same coefficient are streamed together. 

% We compare the four streams using three common quantities, namely the function itself, the
% function's histogram, and an isocontour. Every time a new chunk arrives, we reconstruct the function
% (using the inverse wavelet transform) on the original-resolution grid, and compute the
% root-mean-square error between this reconstructed function and the original function. For the other
% two quantities, the RMSE is replaced by relevant error metrics (see Section~\ref{sec:histogram}
% and~\ref{sec:isocontour} for detailed discussions of these error metrics).
% Figure~\ref{fig:motivation-rmse} shows the results. Note that in all plots, we remove all chunks
% containing purely leading zero bits, as such chunks tend to be compressed away in practice. 

% \begin{figure}[h]
%   \centering
% 	\subcaptionbox{RMSE, left: \emph{plasma}, right: \emph{turbulence}}{
%   {\includegraphics[width=0.48\linewidth]{img/motivation/motivation-psnr-plasma.pdf}}
%   {\includegraphics[width=0.48\linewidth]{img/motivation/motivation-psnr-turbulence.pdf}}}
%   \subcaptionbox{Histogram, left: \emph{plasma}, right: \emph{turbulence}}{
%   {\includegraphics[width=0.48\linewidth]{img/motivation/motivation-histogram-plasma.pdf}}
%   {\includegraphics[width=0.48\linewidth]{img/motivation/motivation-histogram-turbulence.pdf}}}
%   \subcaptionbox{Isocontour, left: \emph{plasma}, right: \emph{turbulence}}{
%   {\includegraphics[width=0.48\linewidth]{img/motivation/motivation-isocontour-plasma.pdf}}
%   {\includegraphics[width=0.48\linewidth]{img/motivation/motivation-isocontour-turbulence.pdf}}}
%   \caption{Comaprisons of the three static streams defined in Section \ref{sec:motivation}. Lower
%   error is better. The plots are truncated to better highlight the differences, without omitting
%   important information. Chunks containing leading zero bits are not plotted. In all cases, \emph{by
%   wavelet norm} performs the best.}
%  	\label{fig:motivation-rmse}
% \end{figure}

% The \emph{by wavelet norm} stream can be seen to consistently performs the best, across all data
% sets and quantities of interest. \emph{by bit plane} works almost as well as \emph{by wavelet norm}
% for function reconstruction and isocontour extraction, but not for histogram computation. The
% reverse is true for \emph{by level}. The \emph{by magnitude} stream performs somewhat better than
% \emph{by level}, except for histogram computation. Note that the our removal of leading zero bits
% penalizes the \emph{by level} stream more than it does the other streams, because in a typical data
% set, most leading-zero chunks exist in fine-resolution subbands (i.e., fine-scale coefficients tend
% to be smaller), which \emph{by level} only accesses later in the stream. The reason \emph{by wavelet
% norm} performs better than the rest is that a higher-ordered bit from a fine-scale coefficient might
% contribute more than a lower-ordered bit from a coarse-scale coefficient does (which \emph{by
% levell} is unable to take advantage of), and vice-versa (which \emph{by bit plane} ignores).

% The fact that \emph{by wavelet norm} outperforms all the other three suggests that performing data
% reduction in both resolution and precision can lead to significant quality improvements. In
% addition, it is possible that different analysis tasks require different types of streams for
% optimal results. This is hinted at by the fact that \emph{by level} underperforms \emph{by bit
% plane} for two of the three tasks, but outperforms it for the third (histogram). It seems likely,
% but uncertain at this point, whether \emph{by wavelet norm} is the best in all cases. To answer this
% question, we will expand our study to include dynamic streams that are optimized specifically for
% each quantity. While these data-dependent streams are unlikely to be realizable in practice, they
% can provide insights into designing practical data streams that improve on the generic \emph{by
% wavelet norm}, by being tailored to the analysis task at hand.

\section{Data-dependent, task-optimized streams}
\label{sec:data_dep_streams}

Each analysis task potentially requires a fundamentally different stream for optimal results. This
section aims to solve the problem of finding the optimal stream, given a data set and
an error metric associated with the analysis task. An error metric is a function
$E(Q(f'),Q(f))$ that returns the distance between its two arguments. $f$ is the original data field
and $f'$ is a reconstructed version of $f$ using a subset of the bits. $Q$ is a quantity of interest
(e.g., histogram, isocontour, etc) derived from $f$ or $f'$. There can be multiple error functions
$E$ for the same $Q$. We choose to use only one error metric with
each quantity, one which is either common, or intuitive and simple while generalizable.
The list of quantity-optimized streams studied in this paper includes \emph{rmse-optimized} (Section
\ref{sec:rmse-optimized}), \emph{gradient-optimized} (Section \ref{sec:gradient}),
\emph{laplacian-optimized} (Section \ref{sec:laplacian}), \emph{histogram-optimized} (Section
\ref{sec:histogram}), and \emph{isocontour-optimized} (Section \ref{sec:isocontour}).

Studying a (data-dependent) quantity-optimized stream is important because such a stream serves both
as a benchmark, and a source of insights for other, more practical streams for the same quantity.
One way to define the ``optimal'' stream for a quantity $Q$ could be the stream that incurs the
minimum error $E$ at every point. However, in trying to realize it, our experience has been that
such a stream does not exist. Assume otherwise that the optimal stream exists, then by definition,
it must be possible to construct this stream using the following greedy algorithm\pavol{the algorithm
may fail to construct optimal stream}: start with a pool
of all chunks (and correspondingly an all-zero $f'$\pavol{zeroed?}, and a presumably very high error $E$), pick
the chunk that when enabled, would minimize $E$, and remove it from the pool. Repeatedly pick and
remove the next chunk that minimizes $E$, until the pool is empty. This algorithm can
encounter a situation in which the next chunk that minimizes the error is on a low-ordered bit plane
of a fine-scale coefficient, which contributes almost nothing to the reconstructed function. The
error is minimized because it is kept approximately constant. In this case, it is actually better to
pick a chunk that increases the error slightly, but otherwise contributes a lot more to the
function. In optimization terms, it is necessary to move in a direction that increases the error to
avoid ending in a local minimum.

The optimal stream for an error metric can also be defined as the stream such that the area bounded
by its plotted error curve and the horizontal axis is the smallest. However, the usefulness of such
a definition is limited in practice, because a stream should be able to be terminated at any point
and still be expected to produce an error as small as possible. Therefore, instead of using this
definition of ``optimality'', we slightly modify the greedy algorithm stated above to avoid the
problem of ending in a local minimum. We start with a pool consisting of all the chunks and an
empty stream, and build this stream back-to-front. In particular, at each step, the chunk whose
removal from the pool has the least impact on the error $E$ is removed and inserted to the beginning
of the current stream. This algorithm solves the problem of unimportant chunks being picked too
early in the original algorithm, because here, chunks picked early are at the end of the
stream, not the beginning.

For our experiments, however, this back-to-front greedy algorithm is too costly. Ignoring
all the steps done in each iteration, this algorithm amounts to a 2-level nested loop running for
$n^2$ iterations, where $n$ is the number of chunks. In 2D, with a $256^2$ data set, a chunk size
that spans $16$ coefficients, and $16$ bits of quantization, $n^2$ would be in the billions, which
we have found to be prohibitively large. We have therefore adopted a simplified version of this
algorithm, where only one pass through the set of $n$ chunks is needed, reducing the number of
iterations from $n^2$ to $n$. The modified algorithm works as follows. We disable (set to zero) a
new chunk $c_i$ in iteration $i$, then compute and record the error $E_i$ due to chunk $c_i$
missing, and enable $c_i$ again at the end of iteration $i$. After repeating the same process for
$n$ iterations, each chunk now has an associated weight, $E_i$. The optimal stream, then, is simply
a sorted list of chunks, in decreasing order of the weights. We observed that this simplified
algorithm brings the running time down from days to minutes, while retaining the same quality of the
output stream. The pseudocode of the algorithm is presented in Algorithm~\ref{alg:greedy}.

\begin{algorithm}[h]
  \caption{Computing a task-optimized stream}
  \begin{algorithmic}[1]
    \Inputs{
			An original function $f$\\
			An unordered set of $n$ chunks $C = \{c_i\}, i\in \{0,\dots,n-1\}$\\
			A quantity of interest $Q$, and a distance function $E$}
		\Initialize{A set of weights $\Gamma = \{\gamma_i\}, i\in \{0,\dots,n-1\}$ }
		\For{each chunk $c_i$}
			\State $c_i \gets 0$
      \State Back-transform $C$ to produce a set of wavelet coefficients $W$
			\State Perform inverse wavelet transform on $W$ to produce $f'$
			\State $\gamma_i \gets E(Q(f'),Q(f))$			
			\State Restore $c_i$
		\EndFor
		\State Sort the $c_i$'s in descending order of $\gamma_i$.
		\Output{The $Q$-optimized stream, which is the sorted $C$}
	\end{algorithmic}
	\label{alg:greedy}
\end{algorithm}

\section{Data-dependent task-optimized streams}
Here we discuss the problem of finding the ``best'' stream for the analysis task at hand. This problem is hard because it is hard to define the notion of ``best''. We will attempt at defining ``best'' in terms of minimizing the error with each bit read. However actually programming the optimization this way will result in a sub-optimal stream because it is necessary to sometimes lose in the short term to then gain more later on. This fact means that we can't optimize for bit budget A and then claim that that solution is also an optimal sub-solution for budget B > A.

Then our next attempt is another $O(n^2)$ scheme where we attempt to maximize the error difference with each bit (Scheme1)

Detail Algorithm for Scheme1.

Yet another way to optimize is the $O(n)$ scheme where we start with every bit and then turn each off and measure the impact on error. Then we sort every bit by its impact (Scheme2).

Detail Algorithm for Scheme2.

We show a figure comparing Scheme1 and Scheme2 on PSNR, histogram, and isocontour for two datasets.

\subsection{Error metrics}
\subsubsection{Function difference}
\subsubsection{Gradient and Laplacian difference}
\subsubsection{Isocontour difference}
\subsubsection{Histogram difference}
Figure 1: show the different histogram metrics
\section{Comparing gradient, Laplacian and rmse streams}
NOTE: for these we use the b-spline 4-4 wavelets which have four vanishing moments so that the reconstructed function is smooth enough for taking derivatives

Exact formulas for computing gradient and laplacian.

Figure 1: gradient difference between gradient and rmse streams

Figure 2: Laplacian difference between Laplacian and rmse

Figure 3: PSNR plot for gradient, laplacian and rmse streams

We conclude that the rmse strea subsumes gradient and laplacian

\section{Comparing histogram and rmse}
We compare different error metric for histogram in two figures:
Figure 1: PSNR plot for all histogram streams using different error metrics.

Figure 2: histogram plots for each metric, at some low bit rate

Figure 3: RMSE error between rmse stream and histogram stream for one data set

Figure 4: Histogram error for the  above two streams for the same data set.

To somewhat quantify the differences of two streams, we define the concept of a stream signature, and present the algorithm to compute a signature.

Algorithm. How to compute a stream's signature.

We show the two signatures for rmse and histogram streams for the same data set.

Figure 5: show signature of the two streams

Figure 6: show signatures for different number of bins

Figure 7: Show that we can stream using the histogram signature to get better histogram than rmse

\section{Comparing isocontour and rmse}
Explain the error metric used for isocontour (number of misclassified voxels + the difference in contour length). The number of misclassified voxels is the main error metric, but we need to add the contour lenngth difference because otherwise chunks that affect less than a pixel cannot be ranked.

To meaningfully compare the isocontour and rmse streams, we devise the hybrid scheme with the idea that when confined to a small enough region, the hybrid stream follows the rmse ordering, while the ordering of regions follows that of the isocontour one (so we can skip regions that do not intersect the contour). We present the detailed algorithm below.

Algorith. How to construct the hybrid stream.

Figure 2: a plot comparing the rmse, isocontour, and hybrid in terms of isocontour error. We observe that the hybrid is close to the isocontour stream.

Figure 2': we compare directly the hybrid and isocontour streams in terms of relative misclassified voxels (between the two of them, not between each and groundtruth).

figure 3: show isocontour rendering for the three streams at some low bit rates where the errors are apparent. Ideally the difference between isocontour and hybrid is not noticeable.

%figure 4:  Suppose we have stream-isocontour and stream-rmse for the same data. We construct stream-hybrid1 by following stream-rmse in terms of ordering of regions, while following stream-isocontour within each region by using its signature. Also construct stream-hybrid2 by following stream-isocontour in terms of ordering of regions and stream-rmse within each region. Yet another hybrid is stream-hybrid3 that follows stream-isocontour both in ordering of regions and within each region. Comparing these streams in terms of isocontour error, we will see stream-isocontour > stream-hybrid3 > stream-hybrid2 > stream-hybrid1 > stream-rmse. This means the concept of signature is meaningful.

Figure 4:
Here, using a synthetic data set (gaussian function) we compare the signature of three isocontour streams at different isovalues, where the derivatives of the function are: low, medium, high. We will observe different orderings (signatures) in each case.

We argue that different isovalues will require different stream signatures, but thanks to the similarity between hybrid and isocontour streams, the rmse stream can be used to extract isocontour too, provided that it is paired with a method to localize the contour (e.g. min-max octrees etc).

Figure 5:
We show that the hybrid and isocontour can diverge somewhat for low-gradient contour. Valerio suggested here we coudl also build a "ramp" dataset at different angles and see if the two diverges more as the ramp become flatter.

We argue that if the gradient is low, some noise bits at the end will make an impact, the isocontour is very sensitive to noise, and is in general not interesting or meaninfgul to extract.
\section{Discussion and Future Work}

{\color{blue}This work addresses one of the biggest challenges in scientific visualization today,
which is the enormous amount of data being generated. We focus on an issue rarely discussed in the
literature, which is the trade-off between two prominent dimensions of data reduction, namely
resolution and precision, when performing analysis tasks. To keep the study tractable while not
compromising the generalizability of results, we avoid multi-parameter tasks such as volume
rendering, in hopes that the fundamental analysis tasks considered here can serve as building blocks
for more complex tasks. Although we are more concerned with the quality of underlying data rather
than visualization outputs, we believe the two often strongly correlate, as demonstrated throughout
the paper. We have also chosen error metrics to measure data quality so that they can be good
proxies for evaluating visualization outputs. The paper focuses on a small set of core tasks, but
the framework is generic and applies to any well-defined metric, and one future direction is to
consider a broader set of tasks.}

We present the first empirical study to demonstrate that combining reduction in precision and
resolution can lead to a significant improvement in data quality for the same data size, and that
different tasks might prefer different resolution-versus-precision trade-offs. For example, while
computing histograms requires high precision, computing derivatives benefits more from higher
resolution, and function reconstruction and isosurface extraction require a proper mix of the two
(see~\Cref{fig:bit-distrib}). We also show that common reduction techniques, e.g., those based on
\slvl and \smag, do not perform well when leading zero bits are removed (to simulate entropy
compression). {\color{red}For each task, the relative ordering of the rate-distortion curves stay
largely the same regardless of data sets, although the gaps between them vary depending on the
smoothness and noisiness of the data. Compared to data-independent streams, signature-based streams
often perform better because they are more adaptive to the data. They are also amenable for
implementation (unlike \sopt), as a signature is negligibly small and thus can be precomputed and
stored during pre-processing. It is also interesting to consider per-block signatures instead of a
global one.}

{\color{red}An important question is whether task- and data-dependent streams provide sufficient
advantages over purely data-independent streams. In practice, data would be used for multiple,
not-necessarily predefined tasks and maintaining multiple streams will likely lead to additional
overheads. Here, we consider \ssig to be the best possible stream that could be realized.
Improvements on \ssig in the resolution-versus-precision space are likely possible, but unlikely to
be significant. Given these assumptions and the fact that \ssig in most cases provides very similar
results to \sbit or \swav, the additional effort (and potential overheads) for task-dependent bit
orderings is unlikely to be beneficial. This leaves a significant gap between the best
data-independent streams and the optimal stream \sopt. Our experiments suggest that the majority of
this difference can be attributed to spatial adaptivity. The prototypical example are isosurfaces
where \sopt can skip all regions that do not effect any portion of the surface. It is worthwhile in
future work to investigate solutions to spatial adaptivity to significantly improve the performance
of data-independent streams.}

This study can be considered only a first step towards a system of solutions that can optimize
storage, network, and I/O bandwidth to suit specific tasks at hand. Ultimately, our results can
guide development of new data layouts and file formats for scientific data. It seems \swav provides
the best all around performance. However, for a file format additional constraints must be
considered, such as disk block sizes, cache coherence, and compression. Furthermore, an ideal format
should allow task-dependent data queries even though task-dependent formats will likely be
restricted to very specific situations. 

Another important consideration is that for fair comparisons we always reconstruct the data at full
resolution using wavelets. However, in practice, processing and memory costs are important, and it
is likely that adaptive representations would be used~\cite{gigavoxels,Gobbetti2008,vdb2013}. In
these cases the error of a given approximation depends not only on the information that is
available, but also on what algorithm and what interpolation method are used. For example,
trilinear interpolation on a coarse grid might provide vastly different results than wavelet
reconstruction on the original mesh. It is possible to specifically construct grids where both
interpolations are equivalent~\cite{weiss}, yet these are not yet implement in standard tools. An
important future research direction will be to understand the implications of the results presented
here on existing toolchains such as VTK.

%%% Local Variables:
%%% mode: latex
%%% TeX-master: "template"
%%% End:

\end{document}
