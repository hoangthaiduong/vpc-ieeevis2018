\section{Derivatives}

In addition to tasks that require an accurate function in the RMSE sense, tasks that concern with
the computation of derivative quantities such as gradient and Laplacian are of fundamental
importance in data analysis. As such, in this section we study streams that minimize gradient and
Laplacian errors. For the experiments in this section, we use the 4-4 B-spline wavelet [CITE] that
has four vanishing moments on each of the analysis and synthesis sides, to ensure that the
reconstructed function is smooth enough for the purpose of taking derivatives.

\subsection{Gradient}

We use the three-point central difference formula for computing partial derivatives in each
direction: $\frac{\partial f}{\partial x}\approx \frac{f(x+1)-f(x-1)}{2}$, and the same for $y$. The
gradient at each point ($\nabla f(x,y)$) is the vector $(\frac{\partial f}{\partial
x},\frac{\partial f}{\partial y})$. The difference between the gradient of a reconstructed function
using $b$ bits per sample $f_b$ and the original function $f$ is defined as $\norm{\nabla f_b-\nabla f}_2$. We use algorithm [CITE] to compute 

Exact formulas for computing gradient and laplacian.

Figure 1: gradient difference between gradient and rmse streams

Figure 2: Laplacian difference between Laplacian and rmse

Figure 3: PSNR plot for gradient, laplacian and rmse streams

We conclude that the rmse strea subsumes gradient and laplacian
