\section{Related work}

\para{Tree-based multiresolution hierarchies.}
Techniques that reduce data in resolution typically build a tree-shape hierarchy over the data. A
very common scheme to generate such a hierarchy is to construct low-resolution copies of the data
from higher resolution ones through filtering or subsampling. Examples include Gaussian and
Laplacian pyramids~\cite{laplacian-pyramid},
mipmaps~\cite{multires_octree1999,interactive-exploration-ct-scans}. Often, the data is stored in
blocks on each resolution level. To save bandwidth, low-resolution blocks can be streamed during
rendering, if the points being queried project to less than a pixel on screen. However, these
methods increase storage requirements and potentially expensive pre-processing, making them
unsuitable for very large data.

Recent multi-resolution techniques save storage by adapting to the data in such a way that different
regions of the data are stored in different resolutions and depending on how ``homogeneous'' is the region.
Fast-varying regions are stored at higher resolution. A very popular approach is sparse voxel
octrees (SVO), pioneered by Crassin \etal~\cite{gigavoxels} and Gobbetti \etal~\cite{Gobbetti2008},
and variations of which are found in~\cite{Fogal-2013-RayGuided,visualization-driven}. Sparsity
comes from the fact that smooth-varying regions are stored at coarser octree levels, which
significantly reduces storage. During rendering, blocks of samples are streamed from an appropriate
resolution, determined by how far the queried samples are from the eye/camera. Beyer
\etal~\cite{large‐scale-volume} give a comprehensive overview of state-of-the-art GPU-based
out-of-core ray-casting techniques.

Beside octrees, other trees such as B+ tree~\cite{vdb2013} and
kd-tree~\cite{fogal-kdtree,in-situ-sampling-particle} can also be used to build a sparse hierarchy.
Alternatively, space-filling curves such as the hierarhical Z curve~\cite{idx2001} or the Hillbert
curve~\cite{mloc} can be used to reorder data samples to form a hierarchy without any filtering
steps or redundant samples, as low-resolution levels are constructed via subsampling. Unfortunately,
subsampling can introduce aliasing.

%\paragraph{\textbf{Transform-based multiresolution hierarchies}}
\para{Transform-based multiresolution hierarchies.}
Other multiresolution approaches reduce data by transform-based compression. For example,
COVRA~\cite{covra2012} constructs an octree of bricks, each of which is further subdivided \pavol{into?} blocks.
Compression is done by learning a sparse representation for the blocks in terms of prototype (basis)
blocks. Similarly, Fout\etal~\cite{hw_dvr2007} transform each block in the volume using the KLT
transform, which produces several classes of partitions (each can be thought of as one resolution
level). They compress the transform coefficients using vector quantization~\cite{vq1992} by
constructing one codebook for each paritition class. Schneider~\etal~\cite{compression_domain2003}
also use vector quantization on transform coefficients, but with a simple Haar-like transform that
separates each block of $2^3$ voxels into one average and seven difference coefficients.

Analogous to the KLT transform in 2D, the Tucker decomposition~\cite{tensor_dvr2015} in
$n$-dimensional space decomposes the input data (stored as a tensor) into $n$ matrices of basis
vectors and one core tensor. Reduction in storage and in transmission bandwidth comes from the fact
that the basis matrices can be downsampled (resulting in a lower resolution representation) and the
core tensor can be rank-truncated (resulting in coarse-scale representations of
features)~\cite{tamresh,tucker-thresholding,multiscale-tensor}. Furthermore, elements in the basis
matrices and the core tensor can be thresholded~\cite{tucker-thresholding} or
quantized~\cite{tamresh,multiscale-tensor}. Tensor decomposition can work for higher dimensional
data and can achieve very high compression ratios. However, the transform step is costly due to it
being data-dependent.

%\paragraph{\textbf{Wavelets}}
\para{Wavelets.}
Transforms that use fixed bases avoid such high computation cost at the expense of slightly less
effective compression. Perhaps the most popular transform that uses a fixed basis is the (discrete)
wavelet transform (DWT). The DWT constructs a hierarchy of resolution levels via low and high
bandpass filters. The transform is recursively applied to the lower resolution band, resulting in a
hierarchy of ``details'' at varying resolution. One benefit of wavelets over redundant
representations such as Laplacian pyramids is that the wavelet transform is merely a change of basis
that like reordering techniques does not increase the data size. Another benefit over adaptive
refinement meshes and other tree-like techniques is that the wavelet basis functions are defined
everywhere in space, requiring no special interpolation rules when given some arbitrary subset of
wavelet coefficients and basis functions. One disadvantage of the wavelet transform is the random
access cost, which is not constant time. However, there has been work to develop acceleration
structures to speed up local queries~\cite{weiss}.

Beside offering a multiresolution decomposition, enabling data streaming with level-of-detail
support, wavelets also offer ample opportunities for compression. As the magnitude of wavelet
coefficients decays rapidly for a typical volume [CITE], they are especially amenable to thresholding
or entropy compression. In the context of storing and visualizing scientific data, wavelets (with
compression) are used in a wide variety of
systems~(\cite{multires_toolkit2003,vapor2007,woodring2011}) and applications (volume rendering
with
level-of-detail~\cite{wavelet-compression-interactive-vis,multires-framework,rapid-compression-volume,interactive-rendering-large-volume,multires-volume-rendering},
turbulence visualization~\cite{treib}, particle visualization~\cite{sph-octree}).

Most wavelet-based techniques employ tiling of wavelet coefficients in individual subands to
facilitate random access and spatial adaptivity in resolution. For example, the VAPOR
toolkit~\cite{vapor2007} incorporates a multiresolution file format based on a wavelets to allow
data analysis on commodity hardware by storing individual tiles in separate files to allow loading
of the region of interest. However, like most multiresolution work, only the resolution control is
leveraged. The precision axis, which can potentially further reduce data transfer, is left
unexplored.

%\paragraph{\textbf{Wavelet coders}}
\para{Wavelet coders.}
Most work that explores the precision axis comes from state-of-the-art coders for wavelet
coefficients in image compression. Wavelet coefficients in corresponding regions across subbands can
be thought of as belonging to a ``tree'', with the root being a single coefficient at the lowest
subband. Thee emmbedded zerotrees (EZW) coder exploits the property that in such trees, ``parent''
coefficients are often larger in magnitude than ``child'' coefficients. It locates trees of wavelet
coefficients that are insignificant with regard to (i.e., less than in magnitude) a threshold. Such
a tree is encoded with one single symbol, resulting in significant compression. The threshold is
typically set at each bit plane, starting from the most significant one. In this way, the data can
be progressively refined in precision during decompression. The SPIHT coder~\cite{spiht1996}
improves on EZW by locating more general types of zero trees~\cite{quantifying-coding-performance}.
SPECK~\cite{speck2004} extends SPIHT to exploit also spatial correlations among nearby coefficients
on the same subband.

%\paragraph{\textbf{Floating-point compression}}
\para{Floating-point compression.}
\newcommand{\zfp}{\textsc{zfp}\xspace}
The \zfp compression scheme~\cite{zfp2014} also encodes transform coefficients by bit plane, in
order of decreasing significance. \zfp partitions the domain---a structured grid---into small (e.g.,
$4 \times 4 \times 4$) independent blocks and thus allows for localized decompression. Moreover,
\zfp supports fixed-rate compression that facilitates random access to the data. Its fast transform
and caching of decompressed data allows it to achieve not only high throughput decompression, but
also fast inline compression. Extensions of \zfp allow it to vary either the bit rate or precision
spatially over the domain, albeit at fixed resolution~\cite{zfp-arc}. Other notable compression
schemes for scientific data include scalar quantization encoding (SQE)~\cite{sqe},
ISABELA~\cite{isabela}, SZ~\cite{sz}, and FPZIP~\cite{fpzip}. The latter three employ prediction and
compress the residuals. ISABELA and SZ perform residual scalar quantization, whereas FPZIP truncates
floats, which can be seen as nonuniform scalar quantization. Similar to FPZIP, the precision-based
level of details (PLoD) scheme proposed in MLOC~\cite{mloc} also truncates floats by dividing a
double-precision number into seven parts, of which the first part contains the first two bytes, and
each of the other six parts contains one byte for additional precision. MLOC includes a
multiresolution scheme based on Hillbert curves, but this scheme (based on resolution) and the PLoD
scheme (based on precision) are exclusive.

%\paragraph{\textbf{Mixing resolution and precision}}
\para{Mixing resolution and precision.}
Schemes that allow progressive data access in both resolution and precision include
SBHP~\cite{sbhp2000} and JPEG2000~\cite{jpeg2000}. Both partition each subband into blocks and code
each block independently, in bit plane order. By interleaving compressed bits across blocks, one can
construct a purely resolution-progressive or a purely precision-progressive stream, or anything in
between. Outside image compression, JPEG2000 has found use in the compression of scientific data. For
example, Woodring \etal~\cite{woodring2011} use JPEG2000 to store floating point data. Since most
JPEG2000 implementations are limited to integer data, the authors apply uniform scalar quantization
to convert floating point data to integer form. Even though JPEG2000 supports varying both
resolution and precision, the authors do not explore this capability but focus only on setting bit
rate. In general, although there exist mechanisms that simultaneously leverage both axes of data
reduction, especially for the purpose of scientific data analysis, they have not been
well-studied.

%\paragraph{\textbf{Error quantification}}
\para{Error quantification.}
Several works have aimed at quantifying the error incurred by data reduction, by
compression or otherwise, for the results of analysis tasks. Baker
\etal~\cite{evaluating-compression-climate} evaluate several compressors (FPZIP~\cite{fpzip},
APAX~\cite{apax}, ISABELA~\cite{isabela}) on ensembles of climate simulation data, using the
root-mean-square z-score (RMSZ) as the error metric. Laney \etal~\cite{compression_sim2013} study
the effects of lossy compression (using FPZIP and APEX) on the Miranda hydrodynamic simulation code,
using physics-based error metrics. Li \etal~\cite{evaluating-efficacy-wavelet} measure the error
incurred by wavelet compression on turbulent-flow data, using root-mean-square error of
radial-enstrophy profiles as the metric. Wang \etal~\cite{statistical-volume-quality} assess the
quality of distorted data using a combination of statistical metrics in the wavelet transform
domain. Etiene \etal have published a series of studies on verification of isosurfaces in
geometrical~\cite{verifiable-isosurface} and topological~\cite{topology-verification-isosurface}
terms, as well as verification of volume-rendered images~\cite{verifying-volume-rendering}. They
focus on order-of-accuracy and convergence analysis, where errors are introduced mostly through
discretization, not necessarily for the purpose of reducing data bandwidth. Finally, the Z-checker
framework~\cite{z-checker} consists of many independent metrics (which are called analysis
``kernels'') to evaluate data quality after lossy compression. The list includes min/max,
distribution, entropy, smoothness, power spectrum, principal component analysis, and
autocorrelation. In general, however, no studies have examined the
resolution-versus-precision tradeoffs, as well as the various orderings of bits in the context of
data analysis and visualization.

%by magnitude streams~\cite{image_compression1992}
% Transfer function adaptive decompression~\cite{tf_decompression2004}

Finally, for surveys of data reduction techniques in general, we refer the readers to the work of
Rodr\'{\i}guez \etal~\cite{state-of-the-art-compressed-volume} and Li \etal~\cite{li2018}.
