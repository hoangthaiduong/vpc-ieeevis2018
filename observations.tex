\section{Observations}
TODO: do we show visualization from the tool?

Introduce the concept of ``signature'' as a way to quantify the relative ordering of bits belonging to different bit planes and subbands (but in the same spatial region).

Figure: example signature of a stream

Algorithm: how to compute signature for a stream

%To demonstrate that this ``signature'' concept is meaningful, we construct ``hybrid'' streams as follows. Suppose we have stream-isocontour and stream-rmse for the same data. We construct stream-hybrid1 by following stream-rmse in terms of ordering of regions, while following stream-isocontour within each region by using its signature. Also construct stream-hybrid2 by following stream-isocontour in terms of ordering of regions and stream-rmse within each region. Yet another hybrid is stream-hybrid3 that follows stream-isocontour both in ordering of regions and within each region. Comparing these streams in terms of isocontour error, we will see stream-isocontour > stream-hybrid3 > stream-hybrid2 > stream-hybrid1 > stream-rmse. This means the concept of signature is meaningful.

\subsection{Comparing gradient, Laplacian and rmse streams}
NOTE: for these we use the b-spline 4-4 wavelets which have four vanishing moments so that the reconstructed function is smooth enough for taking derivatives

Figure 1: gradient difference between gradient and rmse streams

Figure 2: Laplacian difference between Laplacian and rmse

Figure 3: PSNR plot for gradient, laplacian and rmse streams

Figure 4: visual comparison of gradients and laplacian at low bit rate

Figure 5: signature comparing the three streams

We conclude that the rmse strea subsumes gradient and laplacian

\subsection{comparing isocontour and rmse}
Figure 1: show signatures of the two streams

We compute the hybrid stream that follows the isocontour stream in terms of regions, but within each region, follows the rmse stream.

Figure 2: a plot comparing the rmse, isocontour, and hybrid in terms of isocontour error. We observe that the hybrid is close to the isocontour stream.

figure 3: show isocontour rendering for the three streams at some low bit rates where the errors are apparent. Ideally the difference between isocontour and hybrid is not noticeable.

%figure 4:  Suppose we have stream-isocontour and stream-rmse for the same data. We construct stream-hybrid1 by following stream-rmse in terms of ordering of regions, while following stream-isocontour within each region by using its signature. Also construct stream-hybrid2 by following stream-isocontour in terms of ordering of regions and stream-rmse within each region. Yet another hybrid is stream-hybrid3 that follows stream-isocontour both in ordering of regions and within each region. Comparing these streams in terms of isocontour error, we will see stream-isocontour > stream-hybrid3 > stream-hybrid2 > stream-hybrid1 > stream-rmse. This means the concept of signature is meaningful.

Figure 4:
Here, using a synthetic data set (gaussian function) we compare the signature of three isocontour streams at different isovalues, where the derivatives of the function are: low, medium, high. We will observe different orderings (signatures) in each case.

Figure 5: repeat figure3 for the synthetic data set

\subsection{Comparing histogram and rmse}
Figure 1: show signature of the two streams

Figure 2: show comparison of histograms using different number of bins and rmse in terms of PSNR

Figure 3: plot histogram error of rmse and histogram streams and show visually the different histograms at some low bit rate.

Figure 4: show signatures for different number of bins

Figure 5: Show that we can stream using the histogram signature to get better histogram than rmse