\subsection{Isosurface extraction\pavol{update plot text to isosurface}}\label{sec:isocontour}

\begin{figure}[h]
	\centering
	\subcaptionbox{\em pressure, isovalue=0.2}
	{\includegraphics[width=0.48\linewidth]{isocontour/isocontour-optimized-pressure}}
	\subcaptionbox{\em turbulence, isovalue=5}
	{\includegraphics[width=0.48\linewidth]{isocontour/isocontour-optimized-turbulence}}
	\subcaptionbox{\em plasma, isovalue=2}
	{\includegraphics[width=0.48\linewidth]{isocontour/isocontour-optimized-plasma}}
	\subcaptionbox{\em diffusivity, isovalue=-0.05}
	{\includegraphics[width=0.48\linewidth]{isocontour/isocontour-optimized-diffusivity}}
	\caption{Comparison of isocontour errors among streams. Plots are truncated to highlight
	differences without hiding important trends. In all cases, $s_{lvl}$ and $s_{mag}$ perform
	significantly worse than the rest. For \emph{pressure} and \emph{diffusivity}, $s_{iso-sig}
	\approx s_{wav} > s_{bit}$. For \emph{plasma}, there are crossovers between $s_{bit}$ and
	$s_{wav}$. Finally, for \emph{turbulence}, $s_{bit} > s_{wav}\approx
	s_{sig}$.}\label{fig:isocontour-plots}
\end{figure}

Extraction of isosurfaces is an essential task in any visualization and analysis system. For
example, in a combustion simulation, isosurfaces of OH concentration can separate burning and
extinguished regions. In medical imaging, organs can also be separated from the background with
isosurface extraction. Beyond visualization, topological methods such as Reeb graph [CITE] show the
evolution of level sets by identifying equivalence classes of isosurfaces. Among other applications,
Reeb graphs are used in topological simplification and surface segmentation.

In attempting to define an error metric to compare isosurfaces, we have found that the popular
Hausdorff distance [CITE] is not very robust --- it varies drastically with minor changes in the
surfaces. Etiene \etal propose using several geometrical~\cite{verifiable-isosurface} and
topological~\cite{topology-verification-isosurface} metrics for verification of isosurfaces. In this
paper, we have chosen a simple metric that assumes nothing about the surfaces, that is the number of
misclassified voxels in the volume. Given a volume and an isovalue, we assign the label 0 to samples
that are less than or equal to the isovalue, and label 1 to the rest. Samples of the reconstructed
volume are labeled the same way. The distance between the two isosurfaces is then the number of
voxels that are labeled differently between the two volumes.

The metric defined above, however, does not properly measure the importance of a packet, if the
error caused by switching off a packet is too small (in orders of sub-voxel). Obtaining accurate
measurements of such errors is a crucial step in Algorithm~\cref{alg:greedy}. We therefore amend the
error metric based on misclassified voxels, by adding to it the relative difference in area between
the two isosurfaces. This relative difference is computed using the formula $|A_1-A_2|/A(_1)$, where
$A_1$ and $A_2$ are isosurface reas. The normalization reduces the range of this term to $[0, 1]$.
The idea is that when the number of misclassified voxels is less than $1$, the sub-voxel error is
instead captured by this relative difference in surface areas.

With the error metric defined, we compute a data-dependent stream optimized for that metric
($s_{iso-opt}$) and a stream based on its signature ($s_{iso-sig}$) using ~\cref{alg:greedy}
and~\cref{alg:signature}. \cref{fig:isocontour-plots} compares the performances of these two
streams, along with $s_{bit}$, $s_{lvl}$, $s_{wav}$, and $s_{mag}$. Both $s_{lvl}$ and $s_{mag}$
performs poorly, indicating that isosurface extraction favors resolution over precision. There are
crossovers between $s_{bit}$ and $s_{wav}$ for the \emph{plasma} data set. For this data set, the
isosurface is extracted from a region with low gradient, which means the surface is very sensitive
to low-ordered bits (i.e., a slight change in values moves the isosurface by larger distance). For
this reason, the $s_{wav}$ stream initially performs better, as it streams more precision bits. As
$s_{bit}$ acquires enough precision, however, it starts to resolve the fine-scale geometry of the
surface better than $s_{wav}$ does (from~\cref{fig:bit-plane-vs-wavelet-norm-gradient}
in~\cref{sec:gradient}, we learned that $s_{wav}$ tends to reconstruct a smoother function
everywhere). In~\cref{fig:isocontour-surfaces-plasma} we render the surfaces reconstructed by
$s_{bit}$ and $s_{wav}$ at 0.3 bps, which confirms that $s_{bit}$ is able to preserve better the
fine-scale featureas on the surface.

For the \emph{turbulence}

Qualitatively, we studied isosurfaces at specific bit rates for all streams. We were especially
interested at bit rates where the error is not exponentially decaying - the low bit rate isosurfaces
do not resemble the reference full data set isosurfaces. For example, at bit rate 0.4 the error
curves are mostly flat and the gaps between them result in isosurfaces with visible differences
(\Cref{fig:isocontour-surfaces})).

\begin{figure}[h]
	\centering
	\subcaptionbox{\emph{by level} ($s_{lvl}$)}
	{\includegraphics[width=0.31\linewidth]{isocontour/isocontour-level}}
	\subcaptionbox{\emph{by bit plane} ($s_{bit}$)}
	{\includegraphics[width=0.31\linewidth]{isocontour/isocontour-bit-plane}}
	\subcaptionbox{\emph{by wavelet norm} ($s_{wav}$)}
	{\includegraphics[width=0.31\linewidth]{isocontour/isocontour-wavelet-norm}}
	\subcaptionbox{\emph{by magnitude} ($s_{mag}$)}
	{\includegraphics[width=0.31\linewidth]{isocontour/isocontour-magnitude}}
	\subcaptionbox{\emph{by signature} ($s_{iso-sig}$)}
	{\includegraphics[width=0.31\linewidth]{isocontour/isocontour-signature}}
	\subcaptionbox{\emph{reference}}
	{\includegraphics[width=0.31\linewidth]{isocontour/isocontour-groundtruth}}
	\caption{Miranda pressure field isosurfaces at bitrate of 0.4 bps.\pavol{what isovalue?} Both {\em by level}
        and {\em by magnitude} stream exhibit severe blocking artifacts compared to the groundtruth.
        The other streams still show smear artifacts, but the overall structure of the isosurfaces
        is more round and closer to the reference.}
	\label{fig:isocontour-surfaces-pressure}
\end{figure}

\begin{figure}[h]
	\centering
	\subcaptionbox{\emph{by bit plane} ($s_{bit}$)}
	{\includegraphics[width=0.31\linewidth]{isocontour/isocontour2-bit-plane}}
	\subcaptionbox{\emph{by wavelet norm} ($s_{wav}$)}
	{\includegraphics[width=0.31\linewidth]{isocontour/isocontour2-wavelet-norm}}
	\subcaptionbox{\emph{reference}}
	{\includegraphics[width=0.31\linewidth]{isocontour/isocontour2-groundtruth}}
	\caption{\emph{plasma}'s isosurface reconstructed at 0.3 bps.}
	\label{fig:isocontour-surfaces-plasma}
\end{figure}

%We argue that if the gradient is low, some noise bits at the end will make an impact, the
%isocontour is very sensitive to noise, and is in general not interesting or meaninfgul to extract.
