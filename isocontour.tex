\subsection{Isosurface extraction\pavol{update plot text to isosurface}}\label{sec:isocontour}
Another common visualization task is display of boundaries, usually done using isosurfaces. For
example, isosurface of OH concentration can separate burning and extinguished regions in a
combustion simulation or organs can be extracted from medical imaging by threshold on a CT scalars.
Moreover, topological methods such as Reeb Graph is defined by contracting isosurfaces to a point
and building the structure from equivalence classes of isosurfaces. Extraction of isosurfaces is
therefore an essential task in any visualization and analysis system. We thus study the
characteristics of bit streams that minimize errors in the reconstructed isosurfaces and compare
those streams quantitatively and qualitatively.

\pavol{potentially weak point as we do not use any standard error metric here; is the relative
surface area fudge necessary for higher resolution datasets? discuss what other metrics could be
used} As before, we begin by defining an error metric to compare isosurfaces. Commonly used metric
is a Hausdorff distance that measures the shortest path from a point on one surface to the other,
and then taking maximum of all those paths. Unfortunately, the distance is not very robust and can
vary drastically with minor changes in the surfaces. For example, a single perturbation in one of
the two surfaces can result in a large distance even when the surfaces are close to identical.
Therefore, we use a number of misclassified voxels as our distance which can be computed by counting
all voxels that are either in surface $S_1$ and not surface $S_2$ or the other way. \pavol{not sure
if following is needed}We have found that if the error caused by switching off a chunk is too small
(in orders of sub-pixel/sub-voxel), then the importance of the chunk cannot be properly measured. We
therefore amend the error metric by adding to it the relative difference in area between two
isosurfaces. This relative difference is, most of the time, a number between $0$ and $1$, computed
by the formula $|A(S_1)-A(S_2)|/A(S_1)$, where $A(S)$ is the area of an isosurface $S$. The idea is
that when the number number of misclassified voxels is less than $1$, the error -- which is at the
sub-voxel level -- is captured by the relative difference in contour length instead.

With an error metric defined, we can compute an \emph{isosurface-optimized} stream for each data set
in addition to the other streams that are data independent (\Cref{fig:isocontour-plots}). The
hypothesis is resolving an isosurface is primarily affected by the domain resolution. Both
resolution and precision streams are bounded by Nquist limit, a fine isosurface detail may fail to
be resolved either due low resolution or precision. As expected, the {\em isosurface-optimize}
stream performs the best as it has the most freedom when reordering chunks. We observed close to no
difference between {\em by bit plane}, {\em by wavelet norm}, and {\em isosurface signature}
streams. This result is primarily caused by the need for resolution when extracting isosurfaces. For
example, {\em by bit plane} stream will always load full resolution data and vary only precision.

\begin{figure}[t]
	\centering
	\subcaptionbox{pressure}
	{\includegraphics[width=0.48\linewidth]{isocontour/isocontour-optimized-pressure}}
	\subcaptionbox{turbulence}
	{\includegraphics[width=0.48\linewidth]{isocontour/isocontour-optimized-turbulence}}
	\subcaptionbox{plasma}
	{\includegraphics[width=0.48\linewidth]{isocontour/isocontour-optimized-plasma}}
	\subcaptionbox{diffusivity}
	{\includegraphics[width=0.48\linewidth]{isocontour/isocontour-optimized-diffusivity}}
	\caption{Isosurface errors for data-independent, \emph{isosurface-optimized}, and {\em isosurface
	signature} streams. The bit rates are capped to highlight differences among streams. The {\em by
	level} and {\em by magnitude} streams performs worst and the other remaining streams have similar
	performance.}\label{fig:isocontour-plots}
\end{figure}

Qualitatively, we studied isosurfaces at specific bit rates for all streams. We were especially
interested at bit rates where the error is not exponentially decaying - the low bit rate isosurfaces
do not resemble the reference full data set isosurfaces. For example, at bit rate 0.4 the error
curves are mostly flat and the gaps between them result in isosurfaces with visible differences
(\Cref{fig:isocontour-surfaces})).

\begin{figure}[t]
	\centering
	\subcaptionbox{\emph{by level}}
	{\includegraphics[width=0.31\linewidth]{isocontour/isocontour-level}}
	\subcaptionbox{\emph{by bit plane}}
	{\includegraphics[width=0.31\linewidth]{isocontour/isocontour-bit-plane}}
	\subcaptionbox{\emph{by wavelet norm}}
	{\includegraphics[width=0.31\linewidth]{isocontour/isocontour-wavelet-norm}}
	\subcaptionbox{\emph{by magnitude}}
	{\includegraphics[width=0.31\linewidth]{isocontour/isocontour-magnitude}}
	\subcaptionbox{\emph{by signature}}
	{\includegraphics[width=0.31\linewidth]{isocontour/isocontour-signature}}
	\subcaptionbox{\emph{groundtruth}}
	{\includegraphics[width=0.31\linewidth]{isocontour/isocontour-groundtruth}}
	\caption{Miranda pressure field isosurfaces at bitrate of 0.4 bps.\pavol{what isovalue?} Both {\em by level}
        and {\em by magnitude} stream exhibit severe blocking artifacts compared to the groundtruth.
        The other streams still show smear artifacts, but the overall structure of the isosurfaces
        is more round and closer to the reference.}
	\label{fig:isocontour-surfaces}
\end{figure}

\begin{figure}[t]
	\centering
	\subcaptionbox{\emph{reference}}
	{\includegraphics[width=0.31\linewidth]{isocontour/isocontour2-groundtruth}}
	\subcaptionbox{\emph{by bit plane}}
	{\includegraphics[width=0.31\linewidth]{isocontour/isocontour2-bit-plane}}
	\subcaptionbox{\emph{by wavelet norm}}
	{\includegraphics[width=0.31\linewidth]{isocontour/isocontour2-wavelet-norm}}
	\caption{Miranda pressure field isosurfaces at bitrate of 0.4 bps.\pavol{what isovalue?} Both {\em by level}
        and {\em by magnitude} stream exhibit severe blocking artifacts compared to the groundtruth.
        The other streams still show smear artifacts, but the overall structure of the isosurfaces
        is more round and closer to the reference.}
	\label{fig:isocontour-surfaces}
\end{figure}
%Figure 5: We show that the hybrid and isocontour can diverge somewhat for low-gradient contour.
%Valerio suggested here we coudl also build a "ramp" dataset at different angles and see if the two
%diverges more as the ramp become flatter.

%We argue that if the gradient is low, some noise bits at the end will make an impact, the isocontour
%is very sensitive to noise, and is in general not interesting or meaninfgul to extract.

In this section, we looked at isosurface results for all six streams. The {\em by level} and {\em by
magnitude} streams show blocking artifacts compared to the less severe smearing of {\em by wavelet
norm} and {\em by precisio} streams. Therefore, it seems that one of those latter streams combined
with a spatial adaptivity would perform well when isosurface extraction is desired.
