\section{Conclusion}
We presented a study of tradeoff between resolution and precision for commonly used derived
scientific quantities such as RMSE, gradient, Laplacian, histogram, and isosurface.
During this study, we covered gamut of scientific data sets, ranging from simulations
with smooth features or fine small detail to image data with noisy parts.
We showed that one stream type does not fit all analysis tasks, but some streams perform well
in most of them and may be useful in a new file format design.





We started with an evaluation of contemporary techniques for reducing resolution or precision.
The experiments showed that streaming only in resolution or precision is suboptimal for all
tested queries. We thus developed a tractable greedy algorithm for computing adaptive stream order
based on the particular task. The produced stream significantly outperforms any stream that has
a fixed streaming order.

After establishing the greedy algorithm for computing a stream, we focused on each query and
investigated which queries have similar streams. This approach is important because if two streams
are very similar, we can precompute the stream order and apply the same ordering to different
queries.

For example, we learned that the RMSE stream is akin to the gradient stream, and thus only one
needs to be compute. Moreover, since RMSE optimizes for the function, the stream alongside
a good gradient approaches the function itself. In contrast, the gradient ignores the constant offest
of all samples, resulting in an inferior RMSE.


These results should be considered when designing a file format for scientic data. As future work,
we plan to create new file format that will incorporate the stream ordering techniques we presented.
Additionally, such file format needs to support compression and spatial adaptivity to handle large-scale
data and fast queries. In fact, we are investigating if signatures can be used to handle
the spatial adaptial adaptivity. For example, we could extend the signature matrix to the tensor, with
spatial index being the third dimension.


%As we explored the quantities of interest, we showed the RMSE stream is good proxy for
%gradient.
%
%To ground our study, we established baseline streams for each quantity. Given the search
%for optimal stream is exponential in the number of chunks, we described a greedy strategy to obtain
%the baseline stream.
%
%We used the established baseline streams to further investigate streams with goal to identify
%similarities or dissimilarities. We noticed that if a rmse-optimized stream is applied while
%focusing on gradient quantity, the result is almost indistinguishable from the gradient-optimized
%stream. This result suggests, that there is no need to devise heuristic specifically for gradient
%as the rmse one can be used, thus simplifying the implementation.
%
%In contrast, the laplacian can't use the rmse stream. We devised a signature for each stream that
%captures the ordering and can be applied to all the datasets while still outperforming the
%rmse-optimized stream while used for laplacian quantity.
%
%
%RMSE stream can be used for gradient.
%
%
