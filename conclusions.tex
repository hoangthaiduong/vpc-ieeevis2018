\section{Discussion and Future Work}

This work addresses one of the biggest challenges in scientific visualization today, which is the
enormous amount of data being generated. We focus on an issue rarely discussed in the literature,
which is the tradeoffs between two prominent dimensions of data reduction, namely resolution and
precision, when performing analysis tasks. To make the study tractable in a principled way, we avoid
advanced tasks such as volume rendering, the results of which are dependent on too many input
parameters. The fundamental analysis tasks considered here can serve as building blocks for more
complex tasks. Although we are more concerned with the quality of underlying data rather than
visualization outputs, we believe the two often strongly correlate, as demonstrated throughout the
paper. We have also chosen error metrics to measure data quality so that they can be good proxies
for evaluating visualization outputs.

We present the first empirical study to demonstrate that combining reduction in precision and
resolution can lead to a significant improvement in data quality for the same data size, and that
different tasks might prefer different resolution-versus-precision tradeoffs. For example, while
histograms require high precision, computing derivatives benefits more from higher resolution, and
function reconstruction requires a proper mix of the two (see~\Cref{fig:bit-distrib}). We also show
that common reduction techniques, e.g., those based on \slvl and \smag, do not perform well when
leading zero bits are removed (to simulate entropy compression). For brevity we focus on a small set
of core tasks with a single metric for each task. However, the framework is generic and applies to
any well-defined metric and one future direction is to consider a broader set of domain-specific
tasks. 

An important question is whether task- and data-dependent streams provide sufficient advantages over
purely static streams. In practice, data would be used for multiple, not-necessarily predefined tasks
and maintaining multiple streams will likely lead to additional overheads. Here, we consider \sopt
to be the ideal data- and task-dependent stream and \ssig to be the best possible data-independent
stream. In both cases some improvements are likely possible, e.g., a better optimization for \sopt
and different ideas for creating \ssig, but one would not expect these to lead to fundamentally
different results. Given these assumptions and the fact that \ssig in most cases provides very
similar results to \sbit or \swav, the additional effort (and potential overheads) for
task-dependent orderings it unlikely to be beneficial. This leaves a significant gap between the
good static streams, in particular \swav and \sbit, and the optimal stream \sopt. Our experiments
suggest that the majority of this difference can be attributed to spatial adaptivity. The
prototypical example are isosurfaces where \sopt can skip all regions that do not effect any
portion of the surface.

This study can be considered only a first step towards a system of solutions that can optimize
storage, network, and I/O bandwidth to suit specific tasks at hand. Ultimately, our results can
guide development of new data layouts and file formats for scientific data. It seems \swav provides
the best all around performance especially if it could be coupled with a spatially adaptive
hierarchy. However, for a file format additional constraints must be considered, such as disk block
sizes, cache coherence, and compression. Furthermore, an ideal format should allow task-dependent
data queries even though task-dependent formats will likely be restricted to very specific
situations. 

Another important consideration is that to provide a fair comparison we always reconstruct the data
at full resolution. This isolates the question of how much information the data contains from how it
is represented. However, in practice, processing and memory costs are important, and it is likely
that adaptive representations would be used~\cite{gigavoxels,Gobbetti2008,vdb2013}. In these cases
the error of a given approximation depends not only on the information that is available, but also
on what algorithm and what interpolation are used. For example, trilinear interpolation on a coarse
grid might provide vastly different results than wavelet reconstruction on the original mesh. It is
possible to specifically construct grids where both interpolations are equivalent~\cite{weiss}, yet
these are not yet implement in standard tools. An important future research direction will be to
understand the implications of the results presented here on existing toolchains such as VTK.

%%% Local Variables:
%%% mode: latex
%%% TeX-master: "template"
%%% End:
