\section{Conclusion}
We presented a study of trade-off between resolution and precision for commonly used derived
scientific quantities such as RMSE, gradient, laplacian, and histogram. During this study we
covered gamut of scientific data sets, ranging from sharp shockwave with large uniform regions
to smooth and homogeneous viscosity field.

At the beginning we applied commonly used reduction in either resolution or precision and
demonstrated the superior results obtained by interleaving streaming of resolution and precision
bits. Additionally, we showed that different quantities may require vastly different streams.
For example, the stream optimized for RMSE prefers precision but the histogram stream prefers
resolution.

To ground our study, we established baseline streams for each quantity. Given the search
for optimal stream is exponential in the number of chunks, we used greedy strategy to obtain
baseline stream.

We used the established baseline streams to further investigate streams with goal to identify
similarities or dissimilarities. We noticed that if a rmse-optimized stream is applied while
focusing on gradient quantity, the result is almost indistinguishable from the gradient-optimized
stream. This result suggests, that there is no need to devise heuristic specifically for gradient
as the rmse one can be used, thus simplifying the implementation.

In contrast, the laplacian can't use the rmse stream. We devised a signature for each stream that
captures the ordering and can be applied to all the datasets while still outperforming the
rmse-optimized stream while used for laplacian quantity.




As part of future work, we want to extend the analysis to 3D data sets and develop on-disk
representation of the streams.
